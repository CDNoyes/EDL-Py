\documentclass[10pt,a4paper]{article}
\usepackage[latin1]{inputenc}
\usepackage{amsmath, amsthm}
\usepackage{amsfonts}
\usepackage{amssymb}
\usepackage{graphicx}
\newtheorem{theorem}{Theorem}
\author{Connor D. Noyes and Kenneth D. Mease}
\title{Unscented Optimal Control for Mars Entry via Successive Convexification}
\begin{document}
	\maketitle

	\section*{}
	\textbf{This paper presents a numerical approach for solving a class of nonlinear optimal control problems in which the cost functional is a convex function of the state covariance. Using recent advances in successive convexification, the original nonlinear optimal control problem is solved via a convergent sequence of convex subproblems. The unscented transform is used to propagate the system's covariance and fits naturally into a convex approach. An \textit{hp}-adaptive Chebyshev pseudospectral method is used to transcribe each optimal control subproblem into a convex optimization problem and displays benefit relative to typical discretization methods,i.e. the integration methods of Euler and Heun. Numerical results are presented to demonstrate the effectiveness of the approach.}
	
	\section{Introduction}
	% One point to consider: we can swap the independent variable to normalized time, or energy etc and consider a fixed final IV problem. The change of variable increases the nonlinearity but we are iterating over linearizations anyway so it is possible.
	One major challenge in control is to have autonomous systems operate optimally under uncertain conditions. Examples are widespread in aerospace engineering including spacecraft, ascent vehicles, and entry capsules where uncertainty exists in the mass properties, environmental conditions, and vehicle aerodynamics to name only a few, and vehicles resources such as onboard propellant are limited and must therefore be used economically. Standard optimal control techniques consider the performance of a nominal, certain system without regard for the robustness of the solution under perturbations. By explicitly accounting for uncertainty robust optimality can be attained. In order to do so, we consider robust objective functionals which involve expectations of state variables and/or elements of the state covariance. In this work we propose a successive convexification approach to such robust optimal control problems in which the objective functional is convex.

	Reference~\cite{PCE_OCP_Bhattacharya} considered the problem of optimal trajectory generation for systems with probabilistic uncertainty. Polynomial chaos expansions were used to transform stochastic optimal control problems including minimum expectation and minimum covariance problems into deterministic optimal control problems solved by nonlinear programming. This formulation is limited to low dimensional uncertainty spaces due to the factorial increase in the number of terms of the expansion required as the dimensionality grows. In contrast, the unscented transform used in our work scales quadratically. Reference~\cite{OpenLoopUncertain} also explored robust open-loop solutions with a mean-covariance framework based on second order Taylor expansions. Reference~\cite{Desensitized} introduced the closely related desensitized optimal trajectories framework which included both open- and closed-loop formulations. A key difference in the desensitized framework is that it is independent of any probabilistic description of the uncertainty. %but Todd Small's thesis showed they can produce the same results for some weights and that the covariance formulation was more flexible ultimately
	
	Reference~\cite{UnscentedOptimalControl} used the unscented transformation to manage uncertainty in an open-loop optimal control framework, and later extended it to the more general Riemann-Stieljtes optimal control problem \cite{RSOptimalControl} and presented a solution based on hyper-pseudospectral methods. The unscented optimal control problem corresponds to a particular semi-discretization of the optimal control problem and is therefore a special case of the Riemann-Stieljtes formulation. 
			
	Several iterative convex optimization methods have been proposed to solve deterministic optimal control problems with nonlinear dynamics \cite{SeqConProg,SuccConvex1}.	Reference~\cite{SuccConvex1} introduced the successive convexification algorithm based on linearization of nonlinear system dynamics and presented convergence analysis. The algorithm relies on trust regions and virtual controls to ensure solutions to the convexified problem are bounded and feasible. 		
	
	Our work applies the unscented transform to convert the optimal control problem with parametric uncertainty into a higher dimensional deterministic problem. Then, successive convexification is used to solve the resulting unscented optimal control problem. Additionally, closed loop formulations are also considered unlike the purely open loop solutions considered in Refs.~\cite{PCE_OCP_Bhattacharya,OpenLoopUncertain,UnscentedOptimalControl,RSOptimalControl}. In contrast to discretization of continuous dynamics by, e.g., Euler's method, pseudospectral collocation is used to transcribe the problem. The use of pseudospectral methods in convex optimization is rather new \cite{PS_Convex,PS_Convex_ascent} and this appears to be the first use of \textit{hp}-adaptive pseudospectral methods \cite{hp_adapt} which are essential to efficiently solving the large scale problems discussed in this paper. %Although we are motivated by aerospace systems, the method is expected to be more widely applicable.
		
	\section{Problem Statement}
	% We can also consider covariance based constraints with a different convex objective function
	
	The systems under consideration are subject to uncertain continuous nonlinear dynamics 
	\begin{align}
	\dot{x}(t) = f(x(t),u(t),\lambda) \label{eq_dynamics}
	\end{align}
	where $x:[0,t_f]\mapsto \mathbb{R}^n$ is the state trajectory, $u:[0,t_f]\mapsto \mathbb{R}^m$ is the control input, and $\lambda\in\mathbb{R}^p$ is a constant vector representing the parametric uncertainty. Let $z(t)\mathrel{:}=[x(t),\, \lambda]\in\mathbb{R}^{n+p}$. The only state constraints considered here are constraints on the final state, $x(t_f)\in X_f$ where $X_f$ is a convex set. It is further assumed the control is subject to a convex constraint $u(t) \in U \subset \mathbb{R}^m$, and that the parameters are jointly distributed according to a multivariate normal distribution with zero mean and covariance $P_{\lambda}$, i.e. $\lambda \sim \mathcal{N}(0,P_{\lambda})$. The initial state uncertainty is similarly described by $x_0 \sim \mathcal{N}(\bar{x}_0,P_x(0))$ and the total initial covariance of $z(0)$ is then $P(0) = \left[\begin{array}{cc}
		P_{x}(0)& 0_{n\mathrm{x}p} \\
		 0_{p\mathrm{x}n} & P_{\lambda}
		\end{array}\right]$.
	
	For nonlinear systems with strong nonlinearity or significant uncertainty, a linear approach (the state transition matrix approach or directly propagating the covariance differential equations using the Jacobian linearization) may not accurately capture the evolution of the mean and covariance. In contrast, the unscented transform \cite{UT,UKF1} offers an alternative, second order (and thus generally more accurate) estimate of the covariance and also captures second order effects of the uncertainty on the mean. The unscented transform relies on sigma points which are chosen such that they have the same first two moments, i.e. $ (\mathbb{E}[z(0)],\mathbb{V}[z(0)]) = ([\bar{x}_0,\,0],P_0) $, as the initial distribution. These sigma points are then propagated through the nonlinear transformation (here, the integration of nonlinear dynamics) and the transformed points are used to estimate the first two moments of the resulting distribution. There is no unique set of sigma points satisfying these conditions, and there exist both symmetric and asymmetric sets \cite{UT_simplex} that match the first two moments. The symmetric version of the method requires integration of $2(n+p)$ additional trajectories resulting in $2n(n+p)$ dynamic constraints in the optimal control problem. The computational complexity as a function of \textit{n} and \textit{p} is the same as in the linear approach, requiring a constant factor of two more dynamic constraints. 
	
	The sigma points are computed via square root of the covariance matrix scaled by the dimensionality of the problem
	\begin{align}
	&z_0 = \bar{z}(0) \\
	&SS^T = (n+p+\kappa)P(0) \\
	&z_i = \bar{z(0)} \pm S_i 
	\end{align}
	where the Cholesky decomposition is used to efficiently compute \textit{S}, each $S_i$ is understood to be a row of \textit{S}, and $\kappa$ is an additional tuning parameter. Let $\mathcal{S}$ denote the set of sigma points. Each sigma point is propagated forward and the mean and covariance at any point along the trajectory are computed via the linear relationships
	\begin{align}
	&\bar{z}(t) = \mathbb{E}[z(t)] \approx \sum_{i=0}^{2n+2p}w_iz_i(t) \\
	&P(t) = \mathbb{V}[z(t)] \approx \sum_{i=0}^{2n+2p}w_i(z_i(t) - \bar{z}(t))(z_i(t) - \bar{z}(t))^T
	\end{align}
	where $w_i$ are the mean and covariance weights corresponding to the sigma points. The trajectory $x_0(t)$ is referred to as the nominal trajectory. For convenience we define
	\begin{align}
	&X \mathrel{:}= \left[\begin{array}{c}
	x_0\\
	x_1\\
	\vdots\\
	x_{2n+2p}(t)
	\end{array}\right] \\
	&F \mathrel{:}= \left[\begin{array}{c}
		f(x_0(t),u(t),\lambda_0)\\
		f(x_1(t),u(t),\lambda_1)\\
		\vdots \\
		f(x_{2n+2p}(t),u(t),\lambda_{2n+2p}) 
		\end{array}\right] \\
	&\Lambda = \mathrel{:}\left[\begin{array}{c}
			\lambda_0\\
			\lambda_1\\
			\vdots\\
			\lambda_{2n+2p}\end{array}\right] \\
	%&Z(t)\mathrel{:}=[X(t),\, \Lambda] 	\\		
%	&\dot{X}(t) = F(X(t),u(t),\Lambda)	
	\end{align}
	and we can therefore write the dynamics of the overall system as $\dot{X}(t) = F(X(t),u(t),\Lambda)$. Notice that while the uncertain dynamics of Eq.~(\ref{eq_dynamics}) are deterministic for any given realization, the evolution of the higher dimensional system state $X$ is completely deterministic. Thus we pose the following problem:
	\\\\
	\textbf{Problem 1 - Unscented Optimal Control}. \textit{Determine a control function $ u^*(t) $ and state trajectory $ X^*(t) $ that minimize the functional}
		\begin{align}
		J(x,u) = M(\bar{x}(t_f),P(t_f)) + \int_{0}^{t_f}L(\bar{x}(t),u(t),P(t))\mathrm{d}t
		\end{align}
		\textit{subject to the constraints }
		\begin{align}
		&\dot{X}(t) = F(X(t),u(t),\Lambda) \\
		&(X(0),\,\Lambda) = \mathcal{S} \\
		&u(t) \in U \\
		&\bar{x}(t_f) \in X_f \label{eq_endpoint_UT}
		\end{align}
	\textit{where} $ M:\mathbb{R}^{n}\times\mathbb{R}^{(n+p)\times (n+p)}\mapsto \mathbb{R} $ \textit{is the Mayer (terminal) cost,} $L:\mathbb{R}^{n}\times U\times\mathbb{R}^{(n+p)\times (n+p)}\mapsto\mathbb{R}$ \textit{ is the Lagrange (running) cost and both are convex.}
		
	\textit{Remark.} Problem 1 enforces the endpoint constraint on the expected trajectory as represented by Eq.~(\ref{eq_endpoint_UT}) but one may instead choose to enforce it on the nominal trajectory $x^0(t)$.
	
	% Does using UT allow for accounting for control constraints in a closed-loop formulation?
	
	% Problem 2 could be a normal objective function with a covariance constraint instead 
		
%	In particular, we seek trajectories that take advantage of the closed-loop dynamics.
	
%	\subsection{Objective Functions}
	% discussion of possible cost functions here 
%	Trace alone is bad here because it doesn't account for the parameter effects.
% we can also have expected values of convex functions of the state
	
	\section{Solution Methodology}
	Under the stated assumptions, only the nonlinear dynamics render Problem 1 nonconvex. Problem 1 may be solved directly by nonlinear programming (NLP) methods such as sequential quadratic programming (SQP) but such methods do not possess the powerful guarantees of convex programming.
	 We proceed as in Ref.~\cite{SuccConvex1}. The algorithm begins by assuming an initial control history (typically we take the null control $u(t)=0$) and integrating the nonlinear model for each sigma point. The algorithm proceeds by linearizing Eq.~(\ref{eq_dynamics}) around an initial trajectory $ (X^k(t),u^k(t),\Lambda) $, resulting in the following system dynamics for the collection of sigma points
	\begin{align}
	\dot{X}^{k+1} = F(X^{k},u^k) + F_X(X^k,u^k)(X^{k+1}-X^k) + F_u(X^k,u^k)(u^{k+1}-u^k) \label{eq_linearized}
	\end{align}
	which are valid in a neighborhood around $ (X^k(t),u^k(t),\Lambda) $. In order to ensure this validity, we seek a solution $ (X^{k+1}(t),u^{k+1}(t),\Lambda) $ within an ellipsoidal trust region defined by $||X^{k+1}-X^k||_Q <= \delta$. Application of a trust region also prevents the unboundedness of the optimal control problem potentially arising from the use of linearization.\cite{SuccConvex1} 
	
%	each convex subproblem
	
	%\subsection{Extension to Closed Loop Formulations}
	%In an open loop formulation, only the trajectory through the state space can be optimized to reduce the covariance. By introducing linear state feedback, the covariance may be shaped around a fixed trajectory. Problems with an a priori specified feedback gain (constant or scheduled) may be optimized directly using the existing formulation. The optimization of the feedback gains may be performed by convex optimization (as is done here) but other methods are also possible. Simultaneous optimization of the feedback gains and reference trajectory is not convex but we can find the open loop optimal, then optimize the feedback gains for a fixed reference trajectory. Then we alternate between optimizing the reference and feedback gains until convergence. If convergence is not achieved, the open loop optimal trajectory may still be used with gains optimized for that trajectory.
	% Consider optimization of the system with fixed gains computed via LQR or Apollo specificly for Entry. Recompute the gains after each iteration 
		
	
	\subsection{Chebyshev Pseudospectral Transcription}
	The idea behind spectral methods is to approximate solutions (here, state-control trajectories $ (x(t),u(t) $) by a finite sum $x(t) \approx x_N(t) = \sum_{i=0}^{N}b_i\phi_i(t)$ where $b_i$ are the coefficients and $ {\phi_k} $ is a chosen set of basis functions. A wide range of orthogonal polynomial bases and collocation points have been researched \cite{ChebyPS,LegendrePS,RadauPS,GPOPS} and while in theory some are superior to others, in practice it seems that many of them are capable of producing acceptable solutions. For instance, the Legendre and Chebyshev polynomials are often touted as having desirable properties in pseudospectral methods because they correspond to the optimal polynomial approximation in the $L^2([-1,1])$ and $L^\infty([-1,1])$ senses, respectively \cite{Polynomials}, but other methods remain common in the literature as well.
	
	In general, the use of a pseudospectral (PS) method converts the infinite-dimensional optimal control problem into a finite nonlinear programming problem. Due to the underlying convexity of the subproblems posed here, however, the pseudospectral method will result in a convex program that can be solved efficiently and to global optimality by a primal-dual interior point method. The idea to utilize a PS method with convex optimization appears to originate in Ref.~\cite{PS_Convex} where Flipped Radau and Lobatto methods are introduced and used to solve a planetary landing problem. Ref.~\cite{PS_Convex_ascent} used a Gauss PS method to transcribe convex iterations of an optimal ascent trajectory problem. 
	
	Here a Chebyshev PS method, in which solutions are collocated at the extrema of a Chebyshev polynomial, is utilized to transcribe the dynamic constraints. A corresponding Clenshaw-Curtis quadrature scheme is used to estimate the integral in the running cost of each optimal control problem due to the fact it's integration points are exactly the collocation points.\cite{CCQuad} Ref.~\cite{PS_Convex} found that using a PS method produced solutions that better approximated the true continuous dynamics with fewer total collocation points compared to Euler integration. We claim the same is also true here for the cost function approximation, where it is known that equally spaced (i.e. Newton-Coates) quadrature methods converge very slowly while the CC quadrature employed has very favorable convergence properties.
	
	Neither of the applications \cite{PS_Convex,PS_Convex_ascent} to convex optimization referred to another strength of pseudospectral methods: $ hp $-adaptive (mesh based) solutions \cite{GPOPS,hp_adapt}. This well-developed theory can be leveraged here as in the NLP context. A full review is beyond the scope of this paper but the underlying ideas are presented. It is easy to imagine that trajectories exist for which even the best global polynomial approximation cannot provide an adequate solution. In order to improve the quality of the solution, we can instead divide the time horizon into segments and approximate the state on each segment by a comparatively low order polynomial. Refining solutions by allowing the divisions of the time horizon to increase leads to \textit{h}-adaptive methods, while refining the solution by increasing the order of the polynomials on a fixed grid is referred to as a \textit{p}-adaptive method. Combining these two ideas leads to the \textit{hp}-adaptive methods which were originally studied in finite element solutions to fluid mechanics problems \cite{HPAdapt_origin} before being adapted for optimal control. For areas where the solution is smooth, spectral convergence is expected as the order of the approximation is increased.
	
	In the NLP context, the mesh is refined after every solution because the solution is expected to approximate the nonlinear dynamics well. In the successive convexification approach used here, the iterative linearization of the dynamics means that if the trust region is sufficiently large, the solution returned by the convex subproblem may not match the dynamics very closely resulting in large refinement of the mesh at each iteration. Instead, the problem is solved repeatedly on the same mesh until convergence, then refined.
	
	% Mesh refinement uses the true nonlinear dynamics, not the linearized version 
	
	\section{Numerical Results - Application to Mars Entry}
	The technique is applied to an entry vehicle designed to deliver payload to the Martian surface. The longitudinal dynamics are modeled as 
	\begin{align}
	\dot{h} &= v\sin\gamma\\
	\dot{s} &= v\cos\gamma\\
	\dot{v} &= -D - g\sin\gamma \\
	\dot{\gamma} &= \frac{L}{v}\cos\sigma + \left(\frac{v}{h+r_p}-\frac{g}{v}\right)\cos\gamma
	\end{align}
	where \textit{h} is the altitude of the vehicle, \textit{s} is the downrange distance traveled, \textit{v} is the velocity magnitude, and \textit{L} and \textit{D} are lift and drag accelerations given by	
	\begin{align}
	L &= \frac{1}{2}\rho v^2\frac{S}{m}C_l\\
	D &= \frac{1}{2}\rho v^2\frac{S}{m}C_d\\
	\end{align}
	and the atmospheric density $\rho$ is modeled as exponentially decreasing with altitude
	\begin{align}
	\rho = \rho_0e^{-h/h_s} 
	\end{align}
	where $rho_0$ being the density at 0 altitude MOLA, and $h_s$ as an appropriate scale factor.  
	We will use velocity as the independent variable so the full state vector is $x(v) = [h,s,\gamma]$ and define the control to be $u=\cos\sigma$ so that the dynamics may be written
	\begin{align}
	h' &= \dfrac{v\sin\gamma}{-D - g\sin\gamma}\\
	s' &= \dfrac{v\cos\gamma}{-D - g\sin\gamma}\\
%	\dot{v} &= -D - g\sin\gamma \\
	\gamma' &= \dfrac{\frac{L}{v}u + \left(\frac{v}{h+r_p}-\frac{g}{v}\right)\cos\gamma}{-D - g\sin\gamma}
	\end{align} 
	
	The uncertain parameters in this formulation are $C_l,\,C_d,\,\rho_0,\,\mathrm{ and},\,h_s$ (\textit{p}=4) in addition to initial state uncertainty arising from delivery errors. The total number of sigma points is 15 resulting in an extended state space $X \in \mathbb{R}^{45} $. Complete numerical results are forthcoming.
	
	%while the work herein is motivated by aerospace examples we expect the approach to be more broadly applicable 
	
	
	\section{LQ Approach to Covariance Reduction}
	We seek to regulate an LTV system of the form 
	\begin{align}
	\dot{x} = A(t)x(t) + B(t)u(t) 
	\end{align}
	via linear state feedback $u(t) = K(t)x(t)$ while minimizing an objective 
		\begin{align}
	J &= x(T)^TQ_fx(T) + \alpha \cdot \mathrm{trace}(P(T)) \nonumber\\
	&+ \int_{0}^{T}x(t)^TQx(t) + u(t)^TRu(t) + \beta\cdot \mathrm{trace}(P(t)) \partial t
	\end{align}	
	where $P$ is the state covariance matrix, subject to initial condition $P(0) = P_0$ and $\dot{P} = (A+BK)P + P(A+BK)^T$.
	This non-standard objective is non-quadratic due to the variance terms, and the covariance dynamics are nonlinear. However, we will show that this problem can be posed as a standard LQR problem.
		
	\subsection{Rewriting the dynamics as a linear system}
	First, note that the covariance at any time $t$ can be computed via the state transition matrix $P(t) = \Phi(0, t)P_0\Phi(0, t)^T$. As a result, we can replace the covariance dynamics with those of the state transition matrix. The state transition matrix, also called the sensitivity matrix, is governed by the following matrix differential equation and initial condition
	\begin{align}
	&\dot{\Phi} = \frac{\partial \dot{x}}{\partial x}\Phi = (A+BK)\Phi \\
	&\Phi(0) = I. 
	\end{align}
	Now we define $S = \mathrm{vec}(\Phi)$, a vector formed by concatenating the columns of $\Phi$. We further define a "sensitivity" feedback control $V(t) = K(t)\Phi(t)$. The sensitivity dynamics can now be written
		\begin{align}
	&\dot{\Phi} = A\Phi + BV(t)
	\end{align}
%	and in a vector form,
%		\begin{align}
%	&\dot{S} = \tilde{A}S + \tilde{B}U
%	\end{align}
 Finally, by defining $U = \mathrm{vec}(V)$, $z = \begin{bmatrix}
	x \\ S
	\end{bmatrix}$, and  $ v = \begin{bmatrix}
	u \\ U
	\end{bmatrix}$ we have the extended linear system 
		\begin{align}
	\dot{z} = \tilde{A}(t)z(t) + \tilde{B}(t)v(t) 
	\end{align}
	where $\tilde{A}\in \mathbb{R}^{n(n+1) \times n(n+1)}$ is defined 
	\begin{align}
	\tilde{A} = 
	\begin{bmatrix}
	A & & 0 \\
	& \ddots \\
	0 &  & A
	\end{bmatrix}
	\end{align} and $\tilde{B} \in \mathbb{R}^{n(n+1) \times m(n+1)}$ is similarly defined.
	
	\subsection{Rewriting the objective as a quadratic function}
	\begin{theorem}
		The trace of a covariance matrix can be written as a quadratic form of the vectorized sensitivity matrix.
	\end{theorem}
\begin{proof}
	Let $I$ be the n-dimensional identity matrix. We will use the following property of the trace operator:
	\begin{align}
%	\mathrm{trace}(A + B) &= \mathrm{trace}(A) + \mathrm{trace}(B)\\
%	\mathrm{trace}(A \otimes B) &= \mathrm{trace}(A) \mathrm{trace}(B) \\
	\mathrm{trace}(A) &=  \sum_{i=1}^{n} e_i^T Ae_i
%	\mathrm{trace}(ABC) &= \mathrm{trace}(BCA) = \mathrm{trace}(CAB)
	\end{align}
	as well as the following vector sum approach to vectorization of a matrix
	\begin{align}
	\mathrm{vec}(M) = \sum_{i=1}^{n} e_i \otimes Me_i
	\end{align}
	where $\otimes$ denotes the Kronecker product, and $e_i$ is a unit vector in the $i^{th}$ direction, i.e., column $i$ of $I$. 
%	$P_0$ and $P(t)$ are symmetric, positive semidefinite matrices which allows the following decompositions
%\begin{align}
%P &= \Phi P_0 \Phi^T \\
%P&= \Phi LL^T \Phi^T \\
%P&= CC^T\\
%C & = \Phi L
%%C^TC &= L^T\Phi^T\Phi L \\
%\end{align}
%while the vectorization gives us 
%\begin{align}
%S &= \mathrm{vec}(\Phi) = \sum_{i=1}^{n} e_i \otimes \Phi e_i \\
%%S &= e_i (\Phi e_i)^T
%S^T &=  \sum_{i=1}^{n} e_i^T \otimes e_i^T\Phi^T
%\end{align}
We will also need the mixed-product property of Kronecker multiplication,
\begin{align}
(A\otimes B)(C\otimes D) = (AC)\otimes(BD)
\end{align}
which we can apply once to yield 
\begin{align}
(P_0\otimes I)(\sum_{i=1}^{n} e_i \otimes \Phi e_i) = \sum_{i=1}^{n} (P_0 e_i)\otimes(\Phi e_i)
\end{align}
and a second time
\begin{align}
&(\sum_{i=1}^{n} e_i^T \otimes e_i^T\Phi^T)(P_0\otimes I)(\sum_{i=1}^{n} e_i \otimes \Phi e_i) \\
&= \sum_{i=1}^{n} (e_i^TP_0 e_i)\otimes(e_i^T\Phi^T\Phi e_i) \\
&= \sum_{i=1}^{n} (e_i^T\Phi P_0\Phi^T e_i) % This doesn't seem justified  
\end{align}

Now, we can perform the following manipulations 
\begin{align}
\mathrm{trace}(P) &= \mathrm{trace}(\Phi P_0 \Phi^T) \\
\mathrm{trace}(P) &= \sum_{i=1}^{n} e_i^T (\Phi P_0 \Phi^T)e_i \\
%\mathrm{trace}(P) &= \mathrm{trace}(\Phi^T\Phi P_0 ) = \mathrm{trace}(P_0 \Phi^T\Phi)\\
&= \left(\sum_{i=1}^{n} e_i^T \otimes e_i^T\Phi^T\right)(P_0\otimes I)\left(\sum_{i=1}^{n} e_i \otimes \Phi e_i\right) \\
\mathrm{trace}(P) &= S^T(P_0\otimes I)S \\ 
\mathrm{trace}(P) &= S^TDS %= S^TCS
\end{align}
and the final equality is what we set out to prove. 
\end{proof}

Thus, we are left with the following higher dimensional LQR problem 
\begin{align}
\min_v J = z^T\tilde{Q}_fz + \int z^T\tilde{Q}z + v^T\tilde{R}v\\
	\dot{z} = \tilde{A}(t)z(t) + \tilde{B}(t)v(t)
\end{align}
with the augmented weight matrices
\begin{align}
\tilde{Q}_f &= \begin{bmatrix}
Q_f & 0\\
0 & \alpha\cdot D
\end{bmatrix} \\
\tilde{Q} &= \begin{bmatrix}
Q & 0\\
0 & \beta\cdot D
\end{bmatrix} \\	
\tilde{R} &= \begin{bmatrix}
R & 0\\
0 & E
\end{bmatrix} \\
\end{align}
where all of the weights are the same as before except the new weight matrix $E$. The solution to the original problem is recovered in the limit as $E\to 0$. The solution to this problem is well-known,
\begin{align}
v^*(t) = -\tilde{R}^{-1}\tilde{B}^T\Pi(t)z(t)
\end{align}
from which we recover the optimal time-varying feedback gain $K(t)=-R^{-1}B^T\Pi_n(t)$ where $\Pi_n$ is the upper-left $n\times n$ submatrix of $\Pi$. This differs from a standard LQ solution because $\Pi_n$ is affected by the sensitivity terms. 
%In the same way that $R$ penalizes large control effort
%	\begin{align}
%	P_0 = \begin{bmatrix}
%	P_{11} & P_{12} & \dots & P_{1n} \\
%	P_{21} & P_{22} & \dots& P_{2n}\\
%	\vdots & \vdots & \ddots & \vdots \\
%		P_{n1} & P_{n2} & \dots & P_{nn} 
%	\end{bmatrix} \\
%	C = 
%	\begin{bmatrix}
%	P_{11}I & 2P_{12}I & 2P_{13}I & \dots & 2P_{1n}I \\
%	0 & P_{22}I & 2P_{23}I & \dots & 2P_{2n}I \\
%	0 & 0 & P_{33}I & 2P_{34}I & \dots\\
%	\vdots & \vdots &\vdots & \ddots & \vdots \\
%	0 & 0 & 0 & \dots & P_{nn}I
%	\end{bmatrix}
%	\end{align}
%	Note that $S^TCS$ is a scalar quantity and thus $S^TCS = (S^TCS)^T = S^TC^TS$. It follows that $S^TCS = \frac{1}{2}S^T(C+C^T)S = S^TDS$ and it is easy to verify that $D = P_0 \otimes I$. 
\section{Application to Nonlinear Systems via ASRE}	
	\begin{thebibliography}{1}
%		\bibitem{brockett2012}
%		R. W. Brockett. ``Notes on the control of the Liouville equation'' In P. Cannarsa and J. M.
%		Coron, editors, Control of Partial Differential Equations, pages 101-129. Springer,
%			Berlin-Heidelberg, 2012.
		
%		\bibitem{UncertainOptimalControl}
%		C. Phelps, J.O. Royset, and Q. Gong, ``Optimal Control of Uncertain Systems Using Sample Average Approximations," SIAM Journal on Control and Optimization, 2016.
		
		\bibitem{PCE_OCP_Bhattacharya}
		 Fisher, J., Bhattacharya, R., ``Optimal Trajectory Generation with Probabilistic System Uncertainty Using Polynomial Chaos," ASME Journal of Dynamic Systems Measurement and Control, 2011.
		
		\bibitem{OpenLoopUncertain}
		Darlington, J., et al. ``Decreasing the sensitivity of open-loop optimal solutions in decision making under uncertainty." European Journal of Operational Research 121.2 (2000): 343-362.
		
		\bibitem{Desensitized}
		Seywald, H., and Kumar, R.R., ``Desensitized optimal trajectories." Spaceflight mechanics 1996 (1996): 103-115.
		
		\bibitem{UnscentedOptimalControl}
		Ross, I.M., Proulx, R.J., and Karpenko, M., ``Unscented Optimal Control for Space Flight" 24th International Symposium on Space Flight Dynamics (ISSFD). 2014.
		
		\bibitem{RSOptimalControl}
		Ross, I.M., Proulx, R.J., and Karpenko, M., and Gong, Q., ``Riemann-Stieltjes Optimal Control Problems for Uncertain Dynamic Systems," AIAA Journal of Guidance Control and Dynamics, 2015.
		
		\bibitem{BoydConvexBook}
		Boyd, S., and Vandenberghe, L., ``Convex optimization," Cambridge university press, 2004.
		
		\bibitem{Boyd}
		Boyd, S., Crusius, C., and Hansson, A., ``Control Applications of Nonlinear Convex Programming"
		
		\bibitem{UT}
		Julier, S.J., and Uhlmann, J.K., ``A General Method for Approximating Nonlinear Transformations of Probability Distributions" 1996.

		\bibitem{UKF1}
		Julier, S.J., and Uhlmann, J.K., ``A New Extension of the Kalman Filter for Nonlinear Systems" In Int. symp. aerospace/defense sensing, simul. and controls (Vol. 3, No. 26, pp. 182-193), 1997.
		
		\bibitem{UKF2}
		Julier, S.J., and Uhlmann, J.K., ``Unscented filtering and nonlinear estimation." Proceedings of the IEEE, 92(3), 401-422. 2004.
		
		\bibitem{UT_simplex}
		Julier, S.J., and Uhlmann, J.K., ``Reduced Sigma Point Filters for the Propagation of Means and Covariances Through Nonlinear Transformation"
				
		\bibitem{ChebyPS}
		Fariba, F., and Ross, I.M., ``Direct trajectory optimization by a Chebyshev pseudospectral method." Journal of Guidance, Control, and Dynamics 25.1 (2002): 160-166.
		
		\bibitem{LegendrePS}
		Gamal, E., Kazemi, M.A., and Mohsen Razzaghi, M., ``The pseudospectral Legendre method for discretizing optimal control problems." IEEE transactions on Automatic Control 40.10 (1995): 1793-1796.
		
		\bibitem{RadauPS}
		Garg, D., et al. ``Direct trajectory optimization and costate estimation of finite-horizon and infinite-horizon optimal control problems using a Radau pseudospectral method." Computational Optimization and Applications 49.2 (2011): 335-358.
		
		\bibitem{GPOPS}		
		Rao, A.V., et al. ``Algorithm 902: Gpops, a matlab software for solving multiple-phase optimal control problems using the gauss pseudospectral method." ACM Transactions on Mathematical Software (TOMS) 37.2 (2010): 22.				
				
		\bibitem{PS_Convex}
		Sagliano, M., ``Pseudospectral Convex Optimization for Powered Descent and Landing," Journal of Guidance, Control, and Dynamics, 2017.
		
		\bibitem{PS_Convex_ascent}
		Cheng, X., Li, H., and Zhang, R. ``Efficient ascent trajectory optimization using convex models based on the newton-kantorovich/pseudospectral approach". Aerospace Science and Technology, 2017.
		
		\bibitem{hp_adapt}
		Darby, Christopher L., William W. Hager, and Anil V. Rao. ``An hp-adaptive pseudospectral method for solving optimal control problems." Optimal Control Applications and Methods 32.4 (2011): 476-502.
		
		\bibitem{SeqConProg}
		Dinh, Q.T., and Diehl, M., ``Local Convergence of Sequential Convex Programming for Nonconvex Optimization"
		
		\bibitem{SuccConvex1}
		Y. Mao, M. Szmuk, and B. Acikmese, ``Successive Convexification of Non-Convex Optimal Control Problems and Its Convergence Properties," 2017.
		
%		\bibitem{SuccConvex2}
%		Y. Mao, D. Dueri, M. Szmuk, and B. Acikmese, ``Successive Convexification of Non-Convex Optimal Control Problems with State Constraints," 2017.
		
		\bibitem{CCQuad}
		Gentleman, W.M., ``Algorithm 424: Clenshaw-Curtis quadrature [D1]"
		
		\bibitem{HPAdapt_origin}
		Devloo, P.R.B. ``H-p adaptive finite-element method for steady compressible flow," University of Texas, Austin, TX, United States. 1987.
		
		\bibitem{CCQuadCompare}
		Trefethen, L.N., ``Is Gauss Quadrature Better Than Clenshaw-Curtis?" Society for Industrial and Applied Mathematics, 2008.
		
		\bibitem{Polynomials}
		Boyd, J.P., and Petschek, R., ``The relationships between Chebyshev, Legendre and Jacobi polynomials: the generic superiority of Chebyshev polynomials and three important exceptions," Journal of Scientific Computing 59.1 (2014): 1-27.
		
	\end{thebibliography}
\end{document}