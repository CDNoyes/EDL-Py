\chapter{Conclusions}\label{Ch:Conclusions}

%In linear systems subject to Gaussian uncertainties, the covariance matrix evolves independently of the mean trajectory, and must be shaped by feedback control. The natural nonlinearities of the entry problem, particularly the aerodynamic accelerations, which are quadratic in velocity and approximately exponential in altitude, result in drastically different evolution of the state covariance matrix along different trajectories through the atmosphere. This allows for the possibility of open-loop covariance shaping. Introducing a feedback control allows for even greater shaping of the state covariance as well as an additional nonlinearity - saturation. 

In Chapters~\ref{Ch:GuidanceStrategy}-\ref{Ch:SRL_EDL} we have examined the feasibility of designing a reference trajectory that, in closed-loop reference trajectory-based longitudinal entry guidance, yields the robust performance required for higher elevation landing than achieved to date. The entry guidance problem is posed as a robust optimal control problem with uncertain initial state and parameters for the closed-loop dynamics. The performance index is a weighted combination of maximizing mean altitude and minimizing the standard deviations of terminal altitude and downrange distance at the end of the entry phase just prior to the descent and landing phases. The weights on the standard deviations offer an intuitive way to adjust the robust performance. To significantly reduce computation relative to Monte Carlo simulation, the statistics required to evaluate the performance index are estimated with sample trajectories computed via the unscented transform. The resulting large-scale robust optimal control problem is solved by differential dynamic programming. Case studies of MSL- and SRL-class entry vehicles demonstrated both the robust performance of the longitudinal entry guidance and the computational feasibility of the design method. Post-design Monte Carlo simulations verified that the statistics estimated with the unscented transform are sufficiently accurate. An interesting feature of the generated reference bank angle is that, though not imposed a priori, it has margins which result naturally to achieve robustness. While a full quantitative comparison with MSL and Mars 2020 is not possible due to differences in models and assumptions, the results indicate the robust optimal guidance could support higher elevation landings than achieved to date.

In Chapters~\ref{Ch:FuelOptimalPaper} and \ref{Ch:FuelOptimalAssessment} we examined entry guidance for a theoretical mission with a high ballistic coefficient. The primary metric in quantifying performance of the algorithm is the required propellant to achieve pinpoint landing. The target set for the entry guidance algorithm is generated by simulating the performance of the powered descent guidance algorithm. A predictor-corrector strategy is used to track the optimal ignition state. Assessment for a vehicle approximately twice the ballistic coefficient of Mars 2020 demonstrate the guidance is effective in reducing the propellant required; only modest increases in propellant are required despite the presence of large uncertainty in the atmosphere and aerodynamics. 
% SRP Paper Conclusions? Don't really even remember 

\section{Future Work}
There are several key areas in which this research could progress. Concerning the robust optimal entry guidance for current generation missions, the first is consideration of different control laws in order to improve performance.
While entry guidance using a linear control law has been flown on two Mars missions, and also produced compelling results in this dissertation, there remains the prospect of formulating the ROCP using a nonlinear control law to further improve performance. Control methods such as feedback linearization~\cite{FeedbackLinearization}, nonlinear predictive control~\cite{NMPC, JoelController}, and sliding mode control~\cite{SlidingModeObserver,SlidingModeEG1} have been studied in the context of entry guidance. Some of these methods are more directly applicable than others; considerations include differentiability of the control law, the number of free parameters to be chosen or, potentially, optimized, and whether the method requires integration of additional differential equations, especially if the number depends on the number of sigma points. The first consideration is related to the solution via DDP, which requires the second derivatives to exist. Depending on the form of the controller, non-differentiable functions can be approximated by smooth functions, as was done for the saturation function in this work. The latter considerations are to retain tractability of the resulting ROCP, which can quickly grow in scale.
Feedback linearization is a strong candidate because the transformation from the original state to the feedback linearized state is required to be a diffeomorphism, thereby satisfying the differentiability requirement. The method has no additional differential equations. After feedback linearization, the new system has a linear input-output structure, and the new control can be chosen as an affine state feedback, which can be readily optimized by the method presented here.

Another area to consider is further improvements and modifications to the DDP algorithm. The modified DDP algorithm has the capability to optimize the entry flight path angle in addition to the reference control and feedback gains. However, the EFPA has a first-order effect on the total acceleration and heat load. Current vehicles have total acceleration limits of 12-15g; manned missions are expected to have limits near 5g. Incorporating probabilistic state path constraints of the form $\bar{a} + w_a\sigma_a \le a_{\max}$ would make the EPFA optimization more useful, and the weight $w_a$ would allow for a trade between performance and constraint satisfaction. 
%In particular, the parameter optimization capability can be used to treat the entry flight path angle as an optimization variable, but to really make this useful, probabilistic path constraints on the total acceleration and heatload should be considered.

A natural avenue for future work is the inclusion of lateral guidance during phase two and a change to third phase guidance at a fixed velocity. Including the parameters of the lateral deadband in the optimization is difficult due to the discontinuous nature of the command, which is the sign of the bank angle. The lateral guidance commands a bank reversal when the heading error (or another measure of lateral performance, such as crossrange distance) 
\begin{align}
|\psi(v)-\psi^*(v,\state)| > c_0 + c_1v+c_2v^2
\end{align}
where $ \psi^* $ is the heading angle that forms a great circle arc between the current position and the target position and $c_i$ are the coefficients determining the reversal threshold. 
However, it may be possible to consider the full (lateral+longitudinal) state vector, and to model the lateral guidance, but not include the coefficients as part of the optimization process. The sigma point trajectories will react to the lateral guidance, commanding sign changes during the optimization process. Heading alignment guidance can then be modeled since the lateral states are included in the formulation. The heading alignment gain could be treated as an additional optimization variable. Finally, the coefficients can be optimized (or at least tuned carefully) in an outer loop.  

Future avenues for the SRP-based guidance include an efficient method of computing the target set. Convex optimization-based methods to generate constrained controllable sets have been proposed in \cite{SRP_ControllableSets}. As mentioned in Chapter~\ref{Ch:FuelOptimalPaper}, exploring an altitude-based trigger as an alternative to the propellant optimal trigger would make the algorithm significantly less expensive computationally, but it remains to be seen if it will be as effective. Finally, the parametrization used had the advantage of simplicity, but determining an alternative low order parametrization to approximate the structure of velocity-minimizing trajectories is another area of potential improvement. 
 
%%% Local Variables: ***
%%% mode: latex ***
%%% TeX-master: "thesis.tex" ***
%%% End: ***
