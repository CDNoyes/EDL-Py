\chapter{Conclusions}

In linear systems subject to Gaussian uncertainties, the covariance matrix evolves independently of the mean trajectory, and must be shaped by feedback control. The natural nonlinearities of the entry problem, particularly the aerodynamic accelerations, which are quadratic in velocity and approximately exponential in altitude, result in drastically different evolution of the state covariance matrix along different trajectories through the atmosphere. This allows for the possibility of open-loop covariance shaping. Introducing a feedback control allows for even greater shaping of the state covariance as well as an additional nonlinearity - saturation. 

\section{Future Work}
Lateral guidance, and decision to change to third phase guidance, such as heading alignment or deployment position alignment \cite{GuangfeiDissertation}.
Navigational problem? Combined optimal estimation and control, also inclusion of nav error in guidance. While future missions may benefit from improved navigation algorithms and initialization, navigation errors remain an important source of error in assessing guidance performance. 

Nonlinear controller design for improved performance? 

Constrained optimization to make the EPFA optimization more useful, Gload/heating etc. 

Discuss the difficulty of jointly optimizing the lateral corridor parameters due the inherent sign issue. But planning a nominal lateral profile also is not robust.

% Dont know where (not this chapter) but discuss the subtlety of tracking a reference before it is generated. Using the mean means no future points can be taken. Using the nominal, we could conceivably integrate forward (no Feedback), then use that as the reference in a predictive guidance scheme. 

%%% Local Variables: ***
%%% mode: latex ***
%%% TeX-master: "thesis.tex" ***
%%% End: ***
