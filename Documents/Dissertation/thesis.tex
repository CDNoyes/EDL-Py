% This is a template for Ph.D. dissertations in the UCI format.
% 
% All fonts, including those for sub- and superscripts, must be 10
% points or larger.  Recommended sizes are 14-point for chapter
% headings, 12-point for the main body of text and figure/table
% titles, and 10-point for footnotes, sub- and super-scripts, and text
% in figures and tables.
%
% Notes: Add short title to figures, sections, via square brackets,
% e.g. \section[short]{long}.
%
%\documentclass[12pt,fleqn]{ucithesis} % Orig
\documentclass[12pt]{ucithesis} % CDN - centered equations by default? are they supposed to be left justified?
% A few common packages
\usepackage{amsmath}
\usepackage{amssymb} % CDN added, didn't have mathbb command without it
\usepackage{amsthm}
\usepackage{array}
\usepackage{graphicx}
\usepackage{natbib}
\usepackage{relsize}
\usepackage[titletoc]{appendix}

% Some other useful packages
\usepackage{caption}
\usepackage{subcaption}  % \begin{subfigure}...\end{subfigure} within figure
\usepackage{multirow}
\usepackage{tabularx}

% Uncomment the following to attempt to enforce Type 1 or TrueType 
% fonts. ProQuest does not want the type 3 fonts used by default as
% of Dec. 2019 - see 
% https://support.proquest.com/articledetail?id=kA01W000000k9o2SAA . 
% If you are unable to embed fonts such as 'Zapf Dingbats' or 
% 'Symbol', try using raster images (.jpg or .png) instead of vector 
%images (.pdf or .eps).
% \usepackage[T1]{fontenc} 

% plainpages=false fixes the "duplicate ignored" error with page counters
% Set pdfborder to 0 0 0 to disable colored borders around PDF hyperlinks
\usepackage[plainpages=false,pdfborder={0 0 0}]{hyperref}

% Uncomment the following line to use the algorithm package,
% which provides an algorithm environment similar to figure and table
% ("\begin{algorithm}...\end{algorithm}"). A list of algorithms will
% automatically be added in the preliminary pages. Note that you
% probably want a package for the actual code to go with this (e.g.,
% algorithmic).
%\usepackage{algorithm}

% Uncomment the following line to enable Unicode support. This will allow you
% to enter non-ASCII characters (such as accented characters) directly without
% having to use LaTeX's awkward escape syntax (e.g., \'{e})
% NOTE: You may have to install the ucs.sty package for this to work. See:
% http://www.unruh.de/DniQ/latex/unicode/
%\usepackage[utf8x]{inputenc}

% Uncomment the following to avoid "widowing", where page breaks cause
% single lines of paragraphs to float onto the next page (this is not
% a UCI requirement but more of an aesthetic choice).
%\widowpenalty=10000
%\clubpenalty=10000

% Modify or extend these at will.
\newtheorem{theorem}{\textsc{Theorem}}[chapter]
\newtheorem{definition}{\textsc{Definition}}[chapter]
\newtheorem{example}{\textsc{Example}}[chapter]
%%%%% CDN added commands
\newcommand{\state}{\ensuremath{\mathbf{x}}}
\newcommand{\xut}{\ensuremath{\mathbf{z}}}
\newcommand{\xl}{\ensuremath{\mathbf{x}_{\mathrm{lon}}}}
\newcommand{\control}{\ensuremath{\mathbf{u}}}
\newcommand{\ur}{\ensuremath{u_{\mathrm{ref}}}}
\newcommand{\State}{\ensuremath{\mathbf{X}}}
\newcommand{\Control}{\ensuremath{\mathbf{U}}}
\newcommand{\param}{\ensuremath{\mathbf{p}}}
\newcommand{\Param}{\ensuremath{\mathbf{P}}}
\newcommand{\E}[1]{\mathbb{E}\left[#1\right]}
\newcommand{\V}[1]{\mathbb{V}[#1]}
\newcommand{\EUT}[1]{\sum_{i=0}^{2n}\left[#1\right]}
\newcommand{\mean}{\mathbf{m}}
\newcommand{\cov}{C}
\newcommand{\std}{S}
\newcommand{\rtg}{R}
\newcommand{\y}{\ensuremath{\mathbf{y}}}
\newcommand{\dynamics}{\mathbf{f}}
\newcommand\bigzero{\makebox(0,0){\text{\huge0}}}
\newcommand{\normal}{\mathcal{N}}
\newcommand{\trace}{\mathrm{tr}}
\newcommand{\zero}{\mathbf{0}}
\newcommand{\design}{\mathbf{d}}
\newcommand{\dee}{\mathrm{d}}
%\newcommand{\sample}{\ensuremath{\mathbf{z}}}
%\newcommand{\costate}{\mathbf{\lambda}}
%\newcommand{\multiplier}{\mathbf{\nu}}
%\newcommand{\costate}{\mathbf{p}}
%\newcommand{\multiplier}{\mathbf{\lambda}}
%%%%%

% Macros for title, author, abstract, etc.
\thesistitle{Robust Optimal Entry Guidance for High Elevation Mars Landing}

%"Dissertation" for PhD, "Thesis" for master's
\documenttitle{Dissertation}

\degreename{Doctor of Philosophy}

% Use the wording given in the official list of degrees awarded by UCI:
% http://www.rgs.uci.edu/grad/academic/degrees_offered.htm
\degreefield{Mechanical and Aerospace Engineering}

% Your name as it appears on official UCI records.
\authorname{Connor David Noyes}

% Use the full name of each committee member and full title 
% (e.g. Professor/Associate Professor).
\committeechair{Professor Kenneth D. Mease}
\othercommitteemembers
{
  Professor Athanasios Sideris\\
  Professor Tryphon Georgiou
}

\degreeyear{2021}

\copyrightdeclaration
{
  {\copyright} {\Degreeyear} \Authorname
}

% If you have previously published parts of your manuscript, you must list the
% copyright holders; see Section 3.2 of the UCI Thesis and Dissertation Manual.
% Otherwise, this section may be omitted.
% \prepublishedcopyrightdeclaration
% {
% 	Chapter 4 {\copyright} 2003 Springer-Verlag \\
% 	Portion of Chapter 5 {\copyright} 1999 John Wiley \& Sons, Inc. \\
% 	All other materials {\copyright} {\Degreeyear} \Authorname
% }

% The dedication page is optional
% (comment out to exclude).
\dedications
{ 
  To my father Michael and my mother Diane.
}

\acknowledgments
{
  I would first like to thank my advisor Kenneth D. Mease, whose expertise shaped not only the course of my research, but also my entire career. 
  
  I would like to thank my fellow UCI colleagues, especially Drs. Guangfei Duan, Eric Trumbauer, and Alessandro Bombelli, for many fruitful discussions during our time together.
  
  I am also indebted to my committee members, Profs. Sideris and Georgiou, for their expedited review and suggestions. 
  
  I would also like to thank my colleagues at the Jet Propulsion Laboratory for their advice, and for their patience. My experience at JPL profoundly shaped the direction of my research culminating in this dissertation. Special thanks to Joel Benito, whose guidance and mentorship taught me so much in so little time, and whose friendship I am continually grateful for. I am a better engineer for having known you. 
  
  I gratefully acknowledge the Holmes Endowed Fellowship for it financial support. 
  
  Finally, I am immensely grateful to my mother, whose constant support allowed me to pursue this degree.
  
%  You also need to acknowledge any publishers of your previous
%  work who have given you permission to incorporate that work
%  into your dissertation. See Section 3.2 of the UCI Thesis and
%  Dissertation Manual.)
}


% Some custom commands for your list of publications and software.
\newcommand{\mypubentry}[3]{
  \begin{tabular*}{1\textwidth}{@{\extracolsep{\fill}}p{4.5in}r}
    \textbf{#1} & \textbf{#2} \\ 
    \multicolumn{2}{@{\extracolsep{\fill}}p{.95\textwidth}}{#3}\vspace{6pt} \\
  \end{tabular*}
}
\newcommand{\mysoftentry}[3]{
  \begin{tabular*}{1\textwidth}{@{\extracolsep{\fill}}lr}
    \textbf{#1} & \url{#2} \\
    \multicolumn{2}{@{\extracolsep{\fill}}p{.95\textwidth}}
    {\emph{#3}}\vspace{-6pt} \\
  \end{tabular*}
}

% Include, at minimum, a listing of your degrees and educational
% achievements with dates and the school where the degrees were
% earned. This should include the degree currently being
% attained. Other than that it's mostly up to you what to include here
% and how to format it, below is just an example.
%
% CV is required for PhD theses, but not Master's
% comment out to exclude
\curriculumvitae
{

\textbf{EDUCATION}
  
  \begin{tabular*}{1\textwidth}{@{\extracolsep{\fill}}lr}
    \textbf{Doctor of Philosophy in Aerospace Engineering} & \textbf{2021} \\
    \vspace{6pt}
    University of California & \emph{Irvine, CA} \\
    \textbf{Master of Science in Aerospace Engineering} & \textbf{2013} \\
    \vspace{6pt}
    California Polytechnic State University & \emph{San Luis Obispo, CA} \\
    \textbf{Bachelor of Science in Aerospace Engineering} & \textbf{2014} \\
    \vspace{6pt}
    California Polytechnic State University & \emph{San Luis Obispo, CA} \\
  \end{tabular*}

\vspace{12pt}
\textbf{ENGINEERING EXPERIENCE}

  \begin{tabular*}{1\textwidth}{@{\extracolsep{\fill}}lr}
    \textbf{Graduate Intern} & \textbf{2015} \\
    \vspace{6pt}
    Jet Propulsion Laboratory & \emph{Pasadena, California} \\
  \end{tabular*}
  
  \begin{tabular*}{1\textwidth}{@{\extracolsep{\fill}}lr}
      \textbf{Guidance and Control Engineer} & \textbf{2015-present} \\
      \vspace{6pt}
      Jet Propulsion Laboratory & \emph{Pasadena, California} \\
    \end{tabular*}

\vspace{12pt}
\textbf{RESEARCH EXPERIENCE}

  \begin{tabular*}{1\textwidth}{@{\extracolsep{\fill}}lr}
    \textbf{Graduate Research Assistant} & \textbf{2013--2021} \\
    \vspace{6pt}
    University of California, Irvine & \emph{Irvine, California} \\
  \end{tabular*}

\vspace{12pt}
\textbf{TEACHING EXPERIENCE}

  \begin{tabular*}{1\textwidth}{@{\extracolsep{\fill}}lr}
    \textbf{Teaching Assistant} & \textbf{2013} \\
    \vspace{6pt}
    California Polytechnic State University & \emph{San Luis Obispo, CA} \\
  \end{tabular*}
  \begin{tabular*}{1\textwidth}{@{\extracolsep{\fill}}lr}
    \textbf{Teaching Assistant} & \textbf{2015} \\
    \vspace{6pt}
    University of California & \emph{Irvine, CA} \\
  \end{tabular*}
\pagebreak

\textbf{PUBLICATIONS}

  \mypubentry{Mars Entry Guidance for High Elevation via Robust Optimal Control}{2021 (Submitted for Review)}{Journal of Spacecraft and Rockets}

%\vspace{12pt}
%\textbf{REFEREED CONFERENCE PUBLICATIONS}

  \mypubentry{Entry Guidance for Propellant Optimal Powered Descent on Mars}{Aug 2020}{AAS/AIAA Astrodynamics Specialist Conference}
  \mypubentry{A Convex Optimization Approach to Mars Entry Trajectory Updating}{Aug 2018}{AAS/AIAA Astrodynamics Specialist Conference}
  \mypubentry{High Ballistic Coefficient Mars EDL With Supersonic Retropropulsion}{Feb 2017}{AAS Guidance and Control Conference}
  \mypubentry{Sensitivity Analysis and Uncertainty Quantification of a Mars Ascent Vehicle Concept}{Aug 2017}{ASME Verification and Validation Symposium}  
%  \mypubentry{Hybrid propulsion Mars Ascent Vehicle concept flight performance analysis}{Sep 2017}{IEEE Aerospace Conference}
%  \mypubentry{Mars Ascent Vehicle Model Simulation}{Aug 2016}{AIAA/AAS Astrodynamics Specialist Conference}


}

% The abstract was previously limited to a maximum of 350 words, 
% but the UCI manual at https://etd.lib.uci.edu/electronic/td2e#2.2.1.
% currently does not indicate that there is any word limit for the abstract
\thesisabstract
{
  Mars landings at higher elevations than achieved to date are desired for scientific pursuits. The phases of atmospheric flight are entry, descent, and landing. The research reported here concerns the guidance for the entry phase. To support higher elevation landing, entry guidance must deliver the entry vehicle to the required altitude with the required horizontal accuracy at the end of the entry phase. The state-of-the-practice entry guidance cannot both raise the final altitude and achieve the required horizontal accuracy at the end of the entry phase. By formalizing the entry guidance objectives as a robust optimal control problem, we seek both to increase the final altitude and to improve the horizontal accuracy. In this dissertation, we consider only the longitudinal motion and investigate the feasibility of determining a reference trajectory that, in closed-loop reference-trajectory-based guidance, will yield the robust performance required for higher elevation landing. To address robustness, the state variables and uncertain parameters in the entry dynamics are treated as random variables using the unscented transformation to approximate their means and variances and state the performance index in terms of these statistics. Differential dynamic programming is used to solve the robust optimal control problem. Case studies of two different classes of entry vehicle demonstrate both the robust performance of the longitudinal entry guidance and the computational feasibility of the design method.

}


%%% Local Variables: ***
%%% mode: latex ***
%%% TeX-master: "thesis.tex" ***
%%% End: ***


% Add PDF document info fields
\hypersetup{
	pdftitle={\Thesistitle},
	pdfauthor={\Authorname},
	pdfsubject={\Degreefield},
}

% Uncomment the following to have numbered subsubsections (by default
% numbering goes only to subsections).
%\setcounter{secnumdepth}{4}


% Set this to only select a subset of the includes directives below.
% Very handy to speed up compilation if you're working on a certain
% part of your thesis. It conserves page numbers, references, etc.
% even for non-included files.
%\includeonly{chapter5}

\begin{document}

% Preliminary pages are always loaded (TOC, CV, etc.)
\preliminarypages

% Include the different components of your thesis, in separate files.
% Using \include allows you to set \includeonly above.
\chapter{Introduction}
% See Joel page 3 for 3 main difficulties of EDL
%
%
\section{Motivation}

\section{Dissertation Goals}
%\section{Literature Review?}

% Mars entry problem - mission requirements and future stuff

% State of practice - MSL/M2020

% Covariance reduction based entry guidance - also discussion of sensivity based approaches and comparison work. Also terminology: robust OCP versus optim under unc versus...

% Introduction to my approach? 
% State the specific objective(s) of my research 


We demonstrate that, with a reformulation of the objective function, the robust optimal control problem can be solved using differential dynamic programming (DDP) \cite{DDP}, a shooting method originally devised to solve unconstrained nonlinear optimal control problems that has since seen numerous extensions, including to constrained problems \cite{DDP_ControlLimited,HDDP1,HDDP2,DDP_NonlinearConstraints,DDP_InteriorPoint}, and stochastic problems \cite{iLQG, DDP_Stochastic, ozaki_UT,ozaki2020tube}. 
Starting from the control-limited DDP algorithm \cite{DDP_ControlLimited}, we examine a iterated Linear-Quadratic-like simplification \cite{iLQG} that enables solution of large-scale optimal control problems. Next, we extend the algorithm to jointly optimize the static feedback gains along with the reference control.

% TODO: Talk about related work 
%Due to the interest in high elevation landings, guidance algorithms designed to maximize altitude at parachute deploy have been studied using optimal control theory 
%\cite{AltitudeOptimization,AltitudeOptimizationIndirect}. Reference \cite{GuangfeiDissertation} proposed an optimization-based onboard trajectory planning method based on a low-order parametrization designed to achieve high altitude at parachute deploy.
%Typically these papers consider predictive guidance and use parametrizations of the bank angle designed to yield high altitudes. 

% Refences with notes:
% AltitudeUnderUncertainty
% MarsEntryDesensitized % linearized but closed-loop with fixed gain. No consideration of saturation 
% EntryOUU % Does not use linearization, also doesnt solve in a conventional way, maybe remove?
% TrajectoryDesensitization % Desensitized like seywald and kumar, is it entry applied?
% EntryOUUThesis1 % Earth EDL, focuses on footprint computation, only considers open-loop in the reentry problem, and did not conduct monte carlo to confirm
% EntryOUUThesis2 % LQR, minimum control effort objective which stays away from the bounds, angle of attack as control variable. Does perform Monte Carlo(s) to validate improvement. 

Throughout this dissertation, we refer to optimal control formulations that weight a nominal (or mean) optimization objective with minimization of covariance terms as robust optimal control. The nomenclature in the literature also sometimes refer to such problems as uncertain optimal control \cite{PhelpsUncertainOCP}, optimal control under uncertainty. Due to the close relationship between the sensitivity matrix (also called the state transition matrix) and the covariance matrix, we also include desensitized optimal control \cite{Desensitized,TrajectoryDesensitization} under the umbrella of robust optimal control. 
Approaches to entry guidance based on robust optimal control have been studied in \cite{AltitudeUnderUncertainty, EntryOUUThesis1, EntryOUUThesis2, MarsEntryDesensitized, EntryOUU}. Reference~\cite{EntryOUUThesis1} explored the relationship between covariance reduction and sensitivity reduction, and ultimately recommended the covariance formulation for several reasons, including smaller problem size due to the symmetry of the covariance matrix, the ability to incorporate measurements, and simpler formulation and interpretation of variances compared to elements of the sensitivity matrix.

The works most closely related to our approach are Ref.~\cite{AltitudeUnderUncertainty}, where the objective was to maximize a function of the mean and standard deviation of altitude, and Ref.~\cite{MarsEntryDesensitized}, where the objective was to maximize terminal altitude while penalizing sensitivity terms related to variations in the initial state. Compared to Ref.~\cite{AltitudeUnderUncertainty}, our work differs by considering a multi-objective formulation that includes a downrange performance objective rather than strictly focusing on altitude, by considering closed-loop dynamics instead of open-loop, and by using the unscented transform instead of linear covariance propagation. Similarly, Ref.~\cite{MarsEntryDesensitized} also employed linear covariance propagation, but did consider closed-loop performance with fixed feedback gains. Their method made use of sensitivities, rather than statistics, in order to increase robustness to off-nominal situations. Additionally, a single penalty factor was applied to all terminal state sensitivities, including flight path angle. In contrast, we penalize only altitude and downrange deviations, and use a separate penalty factor for each component. An additional difference is that we consider parametric uncertainty in the equations of motion rather than solely uncertainty in the initial state. 

\begin{figure}[h!]
	\centering
	\includegraphics[width=1\textwidth]{Images/RobustTrajectoryOptimization}
	\caption{Comparison of UQ methods in an closed-loop scenario}
	\label{Fig:RobustTrajectoryOpt}
\end{figure}
%The specific objective of this dissertation are to design an entry guidance with the same computational complexity as the state-of-the-practice guidance algorithm while providing superior performance. 

\section{Dissertation Organization}
This remainder of this dissertation is organized as follows: \\
Chapter two discusses the dynamics and models used to derive the guidance algorithm\\
Chapter three describes the entry guidance problem, especially target condition(s)\\
Chapter four outlines the guidance strategy, including posing the problem as a ROCP. Gain optimization (or selection) as well. \\
Chapter five gives an algorithm based on constrained DDP/SLQ to solve the ROCP; Also UQ and reformulations; demonstration of open-loop vs closed loop necessity of UT \\
Chapter six presents the assessment conditions (simulation details, Monte Carlo) Maybe UT-alpha choice should be discussed here?\\
Chapter seven presents an assessment of the guidance algorithm. Choice of alpha parameter. Problems with joint optimization as well (UT cov almost zero but MC cov bad) \\
Chapter eight concludes the dissertation \\

%chapter details two extensions to the DDP algorithm - an augmented lagrangian method for enforcing terminal constraints and a method for optimizing (constant) parameters, such as the feedback gains (lateral corridor params? probably not due to differentiable req) and the initial flight path angle. The gain optimization in particular is important because of the issue with joint optimization of time-varying gains. 

%%% Local Variables: ***
%%% mode: latex ***
%%% TeX-master: "thesis.tex" ***
%%% End: ***

\chapter{Entry Vehicle Dynamics}
\section{Three Degree of Freedom Models}
% Talk about different independent variables, reduced longitudinal state vector 

\section{System Models}
% MarsGRAM vs time-varying mean/std model used in optimization process 


%%% Local Variables: ***
%%% mode: latex ***
%%% TeX-master: "thesis.tex" ***
%%% End: ***

\chapter{Mars Entry Phase Guidance Requirements}\label{Ch:EGConcepts}
% High level ideas, feedback concepts. Target conditions, current missions vs future. PHASES - prebank, phase 2 (range control),phase 3 heading alignment 
Figure~\ref{Fig:M2020Timeline} depicts the EDL timeline for the Mars 2020 mission. The entry phase begins when the vehicle enters the atmosphere at 125 km altitude, and ends when the sequence leading to parachute deployment is commanded. 
\begin{figure}[h!]
	\centering
	\includegraphics[width=0.95\textwidth]{Images/Mars2020Timeline}
	\caption{Entry Descent and Landing timeline for Mars 2020, taken from~\cite{M2020_EDL}.}
	\label{Fig:M2020Timeline}
\end{figure}
%(landing footprint, deployment within safe parachute deployment conditions, sufficient timeline)
As the discussion of MSL in the introduction highlighted, there are several requirements that entry guidance must meet. 
The vehicle must be delivered to safe parachute deployment conditions, generally given as acceptable ranges of dynamic pressure and Mach number, and the deployment must occur high enough to provide timeline margin for terminal descent. The parachute deploy range dispersion limit is flowed back from the landing footprint requirement. MSL determined that the entry phase had to be terminated within 10 km of the parachute deploy target to achieve the desired touchdown ellipse requirement of 25 x 20 km~\cite{MSL_EDL2}. Mars 2020 had the same touchdown requirement~\cite{M2020_EDL}.
Thus, broadly speaking, the goals are to achieve high terminal altitude and accurate terminal range. 
%Constraints - g-load, heating (handled through selection of mean FPA), and parachute. How are we handling chute? Velocity implies, to an extent, certain mach range. 

\section{Entry Phase Termination}\label{Subsec:EntryTermination}
%Termination of the entry phase
Generally speaking, the entry phase of Mars EDL terminates when the sequence leading to parachute deployment begins. For MSL, this sequence was triggered when the onboard estimated vehicle velocity reached a specified value \cite{MSL_EDL2}. Mars 2020 instead triggered at a fixed downrange distance \cite{TriggerComparison2020}. Future missions, especially those intending to land without the use of a parachute, may consider a different function of the vehicle state \cite{LuAdaptiveEDL,NoyesSRP}. The choice of trigger has a strong impact on the terminal distribution. Naturally, triggering based on the onboard estimated velocity reduces the velocity spread down to approximately the spread in velocity knowledge error. Similarly, a range trigger will naturally minimize the range variations at the trigger point, but will result in larger velocity variations (though the increased to the spread in Mach number is not as large due to wind correlations~\cite{TriggerComparison2020}). We remark that designing entry guidance to reduce range errors is not a moot point in the presence of a range trigger. Large errors in range will turn into large variations in the velocity at which the target range is reached, resulting in Mach variations that may exceed parachute qualification limits.
Regardless of the trigger, the vehicle must meet parachute deployment constraints. Given models of the atmospheric density and local speed of sound on Mars, bounds on the acceptable dynamic pressure and Mach number can be converted into a feasible set in altitude-velocity space. See Fig~\ref{Fig:ParachuteBox} for an example using the Mars 2020 values~\cite{M2020_EDL}, including an additional constraint that the deploy altitude is above 6 km.  
\begin{figure}[h!]
	\centering
	\includegraphics[width=0.75\textwidth]{Images/ParachuteBox}
	\caption{Parachute deployment conditions on Mach number and dynamic pressure in altitude-velocity space.}
	\label{Fig:ParachuteBox}
\end{figure}

One key point is that, despite using different triggers to terminate the entry phase, MSL and Mars 2020 used the same design process involving parametrization of the reference control profile that terminates at a fixed velocity. Together, the control profile and the fixed final velocity define the downrange distance that will be used to trigger the termination of the entry phase in a range trigger design. Thus, the reference design is independent of the trigger used to terminate the entry phase. 
% This the main result/conclusion of this chapter. 
Throughout this dissertation it is assumed the entry phase is terminated based on velocity, using the terminal velocity $v_f=460$ associated with a parachute-based landing, corresponding approximately (because Mach also depends on altitude) to deployment at Mach 2. Like MSL and Mars 2020, the role of the longitudinal entry guidance is to command the vertical lift to steer the vehicle to the the target location at the terminal velocity and with sufficient altitude for subsequent landing operations. We will formalize these objectives, and present the remainder of the problem formulation, in Chapter~\ref{Ch:GuidanceStrategy}.

%Need to say somewhere that reducing range dispersions then using a range trigger, will reduce the velocity (and thus Mach) dispersions. 

%Thus, consistent with designing to an appropriate parachute deployment velocity,
 
%In the event of a different trigger, such as one designed to reduce the propellant consumption required to land at the target location, we assume that a similar process can used. For example, the vehicle would carry a fixed amount of propellant which can be used to null a certain amount of velocity; any velocity under that amount is a candidate to end the entry phase. 

% 


%However, this is not the only possibility. We can, for example, target the velocity at which third phase guidance, such as heading alignment \cite{MSL_EDL2} or final position alignment \cite{GuangfeiDissertation}, begins. 

%Discussion of parachute conditions that we aren't modeling in this phase. The parachute conditions could be included in the assessment conditions. Discussion of parachute-free possibilities? 



%%% Local Variables: ***
%%% mode: latex ***
%%% TeX-master: "thesis.tex" ***
%%% End: ***

\chapter{Guidance Strategy}\label{Ch:GuidanceStrategy}
In this chapter we present the guidance strategy, which is to pose the guidance problem as a robust optimal control problem. 
The guidance law for longitudinal range control has an affine feedback structure, consisting of a reference vertical $L/D$ profile and a feedback control law, subject to physical- and safety-related constraints. Thus, to design the guidance means to design the reference vertical $L/D$ and corresponding longitudinal reference trajectory, and the feedback gains. 
%
%The physical constraint is that the available lift is limited, so the most vertical lift that can be achieved is with zero bank angle, or $u=\cos\sigma=1$. The lower bound is dependent on the mission and vehicle characteristics. Physics again limits the magnitude to $u=-1$, but it is often prudent to further limit the lower bound due to the low $L/D$ of typical Mars entry vehicles.

\section{Independent Variable}
Following from the assumption that the entry phase is terminated at a fixed velocity, it is essential to use velocity as the independent variable. If time were used, at best we could constrain the mean velocity at the final time to be equal to the terminal velocity, but there would nevertheless be a distribution of final velocities. Using velocity as the independent variable naturally guarantees that the entire distribution of trajectories will terminate at the same velocity.
Redefining $\state_{\mathrm{lon}} = [h,\,s,\,\gamma]^T$ to remove the velocity state variable, the dynamics with respect to velocity, with $ (\cdot)' $ denoting the derivative with respect to velocity, are 
\begin{align}
	\state_{\mathrm{lon}}'(v,\param) &= \dynamics_v(v,\xl(v,\param),u(v),\param) \label{Eq:DynamicsWRTVel}\\
	&= \begin{bmatrix}
		\frac{\dot{h}}{\dot{v}} \\
		\frac{\dot{s}}{\dot{v}} \\
		\frac{\dot{\gamma}}{\dot{v}} 
	\end{bmatrix}
\end{align}
One issue with this choice, however, is that velocity may not be monotonically decreasing from the entry interface because, at high altitudes, $D\approx0$ and so $\dot{v}=-D-g\sin\gamma$ is dominated by the gravity term until the vehicle descends into denser atmosphere. Because the gains used by MSL and M2020 during range control were interpolated as functions of the velocity magnitude, they had to contend with a similar issue.

Recall that MSL and Mars 2020 had a pre-bank phase prior to range control; this phase ended when the magnitude of vehicle sensed drag acceleration exceeded 1.96 m$/\mathrm{s}^2$ \cite{MSL_EDL2}.
With Martian gravity approximately $3.71$ m/s$^2$, this is sufficient drag to ensure monotonicity for $\gamma$ as steep as $-30^{\circ}$, noting that both MSL and Mars 2020 had entry flight path angles around $-15.5^{\circ}$. 

Because drag is a function of both altitude and velocity, the point at which $D$ reaches its threshold value is not at a single velocity for all realizations of $\param$. Thus, we can consider a very similar solution except that pre-bank ends at a fixed velocity rather than a fixed drag magnitude. Denoting this velocity by $v_0$, this value is essentially the highest velocity for which $D(v_0,\state,\param)>D_{\mathrm{threshold}}$ and is driven by the lowest drag scenario (high altitude, low density, high ballistic coefficient). 
Since the entry uncertainties are generally specified at the entry interface and not at the start of range control, an ensemble of points is propagated from the entry interface to $v_0$, and the mean and covariance matrix at this point become the initial state and covariance used in the robust guidance problem. 

%Given $\bar{\state},\cov_{\state}$ at the EI altitude $\sigma_h=0$, simulate $u = \cos\sigma_{\mathrm{pre}}$ to a ``low" velocity, determine $v_0$, and finally compute $\state_0$ and $\cov_{\state_0}$. 


\section{Feedback Control Form}
As a result of considering uncertainty, the state variables are random variables and the goal is to optimize a distribution of trajectories rather than a single trajectory.
In designing the reference trajectory, the effects of feedback control on the distribution are accounted for. Thus the closed-loop control $u$ consists of both a reference control $\ur$, and a feedback control $\delta u$, in contrast to open-loop design methods where $u=\ur$. Because the control variable is the cosine of an angle, its magnitude must be bounded by one. 
Due to the low lift capability of current generation Mars entry capsules, it may be prudent to further restrict the vehicle from bank angles that orient the lift vector downward, so in our application the lower bound is taken to be zero. The limits apply to the reference control, so the path constraint is
\begin{align}
	0 \le \ur(v) \le 1 \label{Eq:Control_bounds}
\end{align}
which disallows reference bank angle magnitudes greater than $90^\circ$.
The control law is assumed to be to a saturated linear state feedback 
\begin{align}
	u(v,\state_{\mathrm{lon}}) &= \mathrm{sat}_{[0,1]}\left(\frac{\frac{L}{D})_{\mathrm{ref}}\ur(v)}{\frac{L}{D}} + \delta u(v,\state_{\mathrm{lon}})\right) \label{Eq:Control}\\
	\delta u &= k_D\delta D + k_{\gamma}\delta\gamma + k_s\delta s \label{Eq:Feedback}
\end{align}
%TODO: L/D control, reference Eq.3 in MSL design paper, talk about the "strange" form of the open loop component
where we note that, consistent with state-of-the-practice EDL operations on Mars, drag acceleration has been used as a feedback term in place of altitude. 
%This is due to relationship between drag and $s$,
%\begin{align}
%	s = -\int_{E_0}^{E_f}$\frac{\cos\gamma}{D}\mathrm{d}E \approx -\int_{E_0}^{E_f}$\frac{1}{D}\mathrm{d}E
%\end{align}
The saturation function is defined
\begin{align*}
	\mathrm{sat}_{[a,b]}(x) = \left\{\begin{array}{lc}
		a, &  x < a\\
		x, &  a\le x \le b\\
		b, &  b < x
	\end{array} \right. % The period stops a warning about not closing the left 
\end{align*}
The saturation function is required to ensure that, regardless of the value of the reference control $ \ur $, the feedback control $u(v,\state)$ always satisfies the control limits. Since vertical $L/D$ is equal to $\frac{L}{D}u$, the saturation serves the same purpose as the $L/D$ limiter used on MSL and Mars 2020~\cite{MSL_EDL2}. Equation~\eqref{Eq:Control_bounds} limits the reference control, while the saturation function in Eq.~\eqref{Eq:Control} limits the closed-loop control. These bounds need not be equal; ad hoc control margin could be had by imposing tighter bounds in Eq.~\eqref{Eq:Control_bounds} than in the saturation function. However, as pointed out in Ref.~\cite{MSL_EDL2}, some saturation is not necessarily undesirable, so instead we choose the reference limits and closed-loop limits to be equal and allow the optimization process, presented in the following chapter, to determine the robust optimal margin along the trajectory.

In the numerical results presented later, the gains $[k_D, k_{\gamma}, k_s]$ in Eq.~(\ref{Eq:Feedback}) are chosen to be constant values, but they may, in general, be functions of velocity. 
The MSL and Mars 2020 velocity-dependent gains could be used; they are designed, based on linearized dynamics, to fly a trajectory neighboring the reference which ends with the desired range. Our constant gain feedback control is always driving the perturbed trajectory back to the reference. %We will demonstrate that good guidance performance can be achieved.
The state deviations in Eq.~\eqref{Eq:Feedback} are computed with respect to the reference state at the current velocity, e.g., $\delta D(v) = D(v) - D_{\mathrm{ref}}(v)$.
The form of Eq.~\eqref{Eq:Control} may be unfamiliar but notice that when the control is not saturated, Eq.~\eqref{Eq:Control} may be rearranged as
\begin{align}
	\frac{L}{D}u(v,\state_{\mathrm{lon}}) &= \left. \frac{L}{D}\right)_{\mathrm{ref}}\ur(v) + \frac{L}{D}\delta u(v,\state_{\mathrm{lon}}) \label{Eq:Control_rearranged}
\end{align}
Just like the ETPC, the control in Eq.~\eqref{Eq:Control} commands the reference vertical $ L/D $, rather than a reference fraction of the available $ L/D $, which turns out to be a more robust choice in the presence of aerodynamic uncertainty. Compared with the ETPC (see Eq.~(2) in Ref.~\cite{MSL_EDL2}), the difference lies only in the feedback control $\delta u$. Flight path angle is used as a feedback term in place of altitude rate. The rationale is that, when using velocity to interpolate the reference trajectory, only the difference in flight path angle is important. To see this, note $\delta\dot{r}(v) = v\sin\gamma - v\sin\gamma_{\mathrm{ref}} = v(\delta\sin\gamma)$. If the onboard implementation requires that altitude rate be used, we can determine a velocity-varying gain $k_{\dot{r}}$ such that $k_\gamma\delta\gamma \approx k_{\dot{r}}(v)\delta\dot{r}$ as follows
\begin{align}
k_{\dot{r}}(v)\delta\dot{r}(v) &= k_{\gamma}\delta\gamma(v) \\
k_{\dot{r}}(v) &= k_{\gamma}\frac{\delta\gamma(v)}{\delta\dot{r}(v)} \\
\dot{r} &= v\sin\gamma \\
\delta \dot{r} &= v\delta\sin\gamma \\
k_{\dot{r}}(v) &= \frac{k_{\gamma}}{v}\frac{\delta\gamma}{\delta\sin\gamma} \\
\delta\sin\gamma &= 2\cos\frac{\gamma+\gamma_{\mathrm{ref}}}{2}\sin\frac{\gamma-\gamma_{\mathrm{ref}}}{2} \\
\delta\sin\gamma &\approx \cos\gamma_{\mathrm{ref}}(\gamma-\gamma_{\mathrm{ref}}) = \cos\gamma_{\mathrm{ref}}\delta\gamma\label{Eq:FPAGainConvert}\\
k_{\dot{r}}(v) &\approx \frac{k_{\gamma}}{v\cos\gamma_{\mathrm{ref}}(v)}
\end{align}
The error in the approximation in Eq.~\ref{Eq:FPAGainConvert} is small when $\gamma$ is close to $\gamma_{\mathrm{ref}}$. For a large flight path angle deviation of $15^{\circ}$, the error for any $\gamma_{\mathrm{ref}}\in[-20^{\circ},5^{\circ}]$ is less than 3\%. 

\section{Defining the Objective Function}
The first objective is to maximize the mean altitude at the final velocity while minimizing the altitude standard deviation
\begin{equation}
	\max J_h = \bar{h}(v_f) - w_h\sigma_h(v_f) \label{Eq:AltitudeObjective}
\end{equation}
where $w_h\ge0$ is a penalty on the standard deviation. Maximizing Eq.~(\ref{Eq:AltitudeObjective}) for $w_h=0$ results in an optimal mean altitude, while $w_h>0$ maximizes a measure of the low end of the altitude distribution. For example, $w_h=3$ maximizes the 3$\sigma$-low altitude. 

The second objective, consistent with the range control objective, is to minimize the standard deviation of range 
\begin{equation}
	\min J_s = w_s\sigma_s(v_f) \label{Eq:RangeObjective}
\end{equation}
In contrast to Eq.\eqref{Eq:AltitudeObjective}, the mean range is not included in the objective. This is because for altitude, achieving a very tight altitude distribution is not sufficient; the mean altitude must also be high for the low end of the distribution to have sufficient timeline margin. Regardless of the length of the trajectory flown, the goal is to minimize the standard deviation to achieve a tight distribution. While for a given mission the target location on the ground may be fixed, during mission planning the entry point can adjusted to accommodate the optimal trajectory length. 

The overall performance objective is simply the sum of $J_h$ and $J_s$, posed as a minimization problem
\begin{align}
	\min J = -\bar{h}(v_f) + w_h\sigma_h(v_f) + w_s\sigma_s(v_f) \label{Eq:Objective}
\end{align}
Equation~\eqref{Eq:Objective} may also be interpreted as the sum of a performance objective (mean altitude) and a robustness objective (the standard deviations). Regardless of the interpretation, the weights offer a simple way to adjust closed-loop performance. Due to the nonlinear dynamics, the state distribution will not remain normally distributed; nevertheless we assume that standard deviations remain an appropriate measure of the spread of the terminal distribution. That is, that a reduction in standard deviation will lead to a tighter distribution regardless of the shape of the distribution. Either variances or standard deviations may be used in Eq.~(\ref{Eq:Objective}). In practice, appropriate choices of the weights can produce equivalent results, but standard deviations are preferred because they are in the same units as the state variables, and allow for more natural interpretations of the weights. An additional consideration is that the square root is not differentiable at zero, but in practice, the terminal standard deviations are never exactly zero. Additionally, a small term $\epsilon<<1$ can be added to safeguard against this issue in the event the terminal covariance matrix is only positive semidefinite.
%, as might be the case at the initial velocity when considering a certain initial state subject to parametric uncertainty. 
%TODO: Discuss the strangeness of defining a reference trajectory by the closed loop mean? Normally, a reference trajectory is created, and the mean is estimated via MC after simulation. 

\section{The Robust Optimal Guidance Problem}\label{Sec:ROGP}
In summary, the robust optimal guidance problem (ROGP) is to determine $u[v_0,v_f]$, parametrized by $\ur(v)$ and $K(v)$, that minimizes
\begin{align}
	&\min J = -\bar{h}(v_f) + w_h\sigma_h(v_f) + w_s\sigma_s(v_f) \nonumber\\
	&\mathrm{subject\, to }\nonumber \\
	&\xl'(v,\param) = \dynamics_v(v, \xl(v,\param), u(v), \param),\quad
	\xl(v_0,\param) = \state_0(\param) \label{Eq:ROGP}\\
	&\param\sim \normal(\mathbf{0},\cov_{\param}) \nonumber\\
	&0 \le \ur(v) \le 1 \quad \forall\,v\in [v_0, v_f] \nonumber
\end{align}
The robust optimal guidance solution consists of the reference control and the reference trajectory, and the feedback gains. The reference trajectory is defined to be the mean trajectory, $\bar{\state}_{\mathrm{lon}}(v)$, rather than the nominal trajectory, $\xl(v,\zero)$. 

The succinct form of Eq.~\eqref{Eq:Objective} perhaps belies the challenging, highly nonlinear nature of the objective functional, which features multiple integrals over the uncertainty space
\begin{align}
	\min J = &-\int_{\mathbb{R}^{n_p}}h(v_f,\param)\mu(\param)\dee\param \nonumber\\
	&+ w_h\left[\int_{\mathbb{R}^{n_p}}\left(h(v_f,\param) - \bar{h}(v_f)\right)^2\mu(\param)\dee\param\right]^{\frac{1}{2}} \\
	&+ w_s\left[\int_{\mathbb{R}^{n_p}}\left(s(v_f,\param) - \bar{s}(v_f)\right)^2\mu(\param)\dee\param\right]^{\frac{1}{2}}\nonumber
\end{align}
where we recall that $\mu = \normal(\mathbf{0},\cov_{\param})$ is the probability density function of $\param$. The following chapter will discuss the use of the unscented transform to form a tractable, finite-dimensional approximation to the objective functional. Part of the motivation behind using the mean state to define the reference trajectory is that, if the unscented transform estimates the mean range accurately, then by using the mean range as the target range, the mean range error should be very nearly zero in the Monte Carlo assessment results. 

%%% Local Variables: ***
%%% mode: latex ***
%%% TeX-master: "thesis.tex" ***
%%% End: ***

\chapter{Solving the Robust Optimal Guidance Problem}\label{Ch:SolutionMethod}
% UT transform to quantify mean/std of key variables
% Reformulation of the problem as a running cost and with a fixed independent variable 
% SLQ version of Differential Dynamic Programming algorithm for large scale solution 
% Analysis of necessary conditions??

Solving the robust optimal guidance problem presents a number of challenges. In this chapter we first discuss our choice of uncertainty quantification method used to compute the mean and standard deviation of key variables. Next we present the differential dynamic programming algorithm used to numerically solve the guidance problem. Finally, we detail several modifications to the problem to make it amenable to solution via DDP.

\section{Uncertainty Quantification}\label{Sec:UQ}
A key issue in solving the robust optimal guidance problem is computing the expected values and standard deviations in the objective functional and feedback terms. Broadly speaking, uncertainty quantification (UQ) methods trade between accuracy and the amount of computation required. For example, linear covariance propagation is one of the most efficient UQ methods for computing the first two probability moments, but its accuracy depends on the nonlinearity of the system dynamics. At the other extreme, Monte Carlo simulation can estimate higher order moments to arbitrary accuracy, but may require a huge number of samples in order to do so. Since UQ will be performed at each optimal control solver iteration, the method chosen must strike a careful balance these two aspects. For very fast but inaccurate methods, the solution may not perform as expected in a higher fidelity UQ, such as Monte Carlo simulation, and the benefit of the approach may be diminished or lost entirely. On the other hand, accurate methods will result in very long solution times. In this work, the unscented transform (UT) \cite{UT1997} is chosen to compute the required statistics. The following subsections review linear covariance and unscented transform methods. Then, via numerical example, we demonstrate that the saturation nonlinearity, introduced by the feedback control formulation Eq.~\eqref{Eq:Control}, is more accurately accounted for using the unscented transform. 

%The UT is preferable to linear covariance propagation in our application because the sigma points, described in a following subsection, are propagated through the full nonlinear, saturated dynamics, and are thus able to capture their effects on the distribution more accurately than linearization.

\subsection{Linear Covariance}
In the linear covariance methodology, the entry dynamics are linearized around a nominal state-control trajectory, and the mean trajectory is assumed to be equal to the nominal trajectory, i.e., $ \E{\state(v,\param)} \approx \state(v,\mathbf{0}) $. Compared with Eq.\eqref{Eq:TaylorExp}, we see this neglects a term related to the product of the covariance matrix and the trajectory curvature. 
The covariance matrix is propagated by integrating the Lyapunov differential equation
\begin{align}
	\dot{\cov}_{\state}(t) &= A(v)\cov_{\state}(v) + \cov_{\state}(v)A^T(v) \\
	A(v) &= \frac{\partial \dynamics_v}{\partial\state}\bigg\rvert_{x(v,\mathbf{0})}
\end{align}
subject to the initial condition $\cov_{\state}(v_0) = \cov_{\state_0}$. Note that the linearization around the nominal trajectory uses dynamics with respect to velocity, Eq.~\eqref{Eq:DynamicsWRTVel}. 
%TODO: clraify the notation, parameters are include in the state vector and thus the derivatives as well. 

\subsection{Unscented Transform}\label{Sub:UT}
The unscented transform is a method to approximate the first two moments of a nonlinear transformation of a probability distribution. Consider a scalar quantity $q\in\mathbb{R}$ resulting from a nonlinear transformation $q = F(\y)$ of a random vector $\y\in\mathbb{R}^n$ with known mean $ \bar{\y} $ and covariance $ \cov_{\y} $. A set of $2n+1$ sigma points and associated weights are computed 
\begin{align*}
	\y_0 &= \bar{\y} \\
	\y_i &=  \bar{\y} + \left(\sqrt{(\alpha+n) \cov_{\y}}\right)_i,\, \,i=1,...,n \\
	\y_{i+n} &=  \bar{\y} - \left(\sqrt{(\alpha+n)\cov_{\y}}\right)_i, \, \,i=1,...,n\\
	w_0 &= \frac{\alpha}{\alpha+n} \\
	w_i &= w_{i+n} = \frac{1}{2(\alpha+n)}, \, \,i=1,...,n
\end{align*}
where $\alpha$ is a scaling parameter, and $\left(\sqrt{(\alpha+n)\cov_{\y}}\right)_i$ is the $i^{\mathrm{th}}$ column of the matrix square root of $(\alpha+n) \cov_{\y}$. The sigma points are then mapped through the transformation
\begin{align}
	q_i = F(\y_i),\;\;i=0,...,2n
\end{align}
and finally, the mean and variance are estimated using the weights and transformed sigma points
\begin{align*}
	\bar{q} &\approx \sum_{i=0}^{2n}w_iq_i\\
	\sigma_{q} &\approx \left(\sum_{i=0}^{2n}w_i\left(q_i - \bar{q}\right)^2\right)^{\frac{1}{2}}
\end{align*}
In our application, $\y=\param$, the nonlinear transformation is the integration of the longitudinal entry dynamics to the terminal velocity, and the quantities of interest are downrange distance and altitude. For a given covariance matrix and a chosen value of $\alpha$, the ensemble of sigma points may be collected into a single extended vector, and the robust guidance problem is now a deterministic problem in the extended state space.
\begin{align}
	X = \begin{pmatrix}
	\state^{(0)}\\
	\vdots\\
	\state^{(2n)}
	\end{pmatrix}
\end{align}
When applying the unscented transform, the scaling parameter $\alpha$ is important in precisely estimating the statistics. For any value of $\alpha$, the sigma point distribution has the same mean and covariance as the initial distribution. Increasing the value of $\alpha$ places the sigma points $\y_i$ further from the nominal sigma point $\y_0$ and reduces their weight. Using a small value of $\alpha$ results in sigma points with only small deviations from the nominal, and for the entry guidance problem in this paper, the effects of the controller and saturation nonlinearity will not be accurately quantified. In our numerical studies, it appears that no single value of $\alpha$ minimizes estimation errors for all control profiles and all quantities of interest. However, it is important to recall that estimating statistics is not the purpose of the proposed trajectory optimization. So long as the reference trajectories designed using the UT confer benefits in Monte Carlo simulation, the UT statistics are sufficiently accurate, as will be demonstrated by the results of the numerical assessment. 

\subsection{Linear Covariance vs Unscented Transform Example}\label{Subsec:UQExample}
In this subsection we present a numerical example demonstrating the shortcoming of linear covariance propagation for our application to closed-loop entry guidance.
We simulate the performance of an MSL class entry vehicle using the feedback control, Eq.~\eqref{Eq:Control}. The example considers a single, non-optimized reference control and trajectory, and two sets of static feedback gains, $K_1=[0,0,0]^T$ and $K_2=[0.0725, -0.025, -0.004]^T$. $K_1$ corresponds to the open-loop case. The ballistic coefficient at the initial velocity is $\beta=120\,\mathrm{kg/m}^3$ and $L/D = 0.24$. The reference control is a linear-in-velocity ramp from 0.35 to 1. Additional details such as initial state and covariance are given in Chapter~\ref{Ch:AssessmentConditions}.
%TODO: finish giving the problem data,  uncertainties,

The terminal statistics are computed with three different UQ methods: linear covariance propagation (LC), unscented transform (UT), and Monte Carlo (MC). While the MC statistics are themselves estimates of the true mean and variance, accurate estimates can be by using a sufficiently large sample size. Thus they are the values against which statistics computed by LC and UT are compared. The UT scaling parameter $\alpha=15$ is used; sensitivity to this parameter is discussed below. Table~\ref{Table:UQCompareOpenLoop} presents the terminal statistics for the results using $K_1$; in this open-loop problem both LC and UT perform well. LC underestimates mean altitude by 1.2 km and altitude standard deviation by 200 m, underestimates mean range by 1 km, and overestimates range standard deviation by 1.1 km. In contrast, the UT overestimates the mean altitude by 160 m, the altitude standard deviation by 65 m, underestimates mean range by 800 m, and overestimates range standard deviation by 180 m. While the UT estimates a better in each variable, the LC estimates could be considered sufficient for the purposes of optimization. 

\begin{table}[h!]
	\centering
	\includegraphics[width=1\textwidth]{Images/UQExample_Open}
	\caption{Comparison of UQ methods in an open-loop scenario}
	\label{Table:UQCompareOpenLoop}
\end{table}
\begin{table}[h!]
	\centering
	\includegraphics[width=1\textwidth]{Images/UQExample_Closed}
	\caption{Comparison of UQ methods in an closed-loop scenario}
	\label{Table:UQCompareClosedLoop}
\end{table}

Table~\ref{Table:UQCompareClosedLoop} presents the terminal statistics for the results using $K_2$. Linear propagation is listed twice; the first case linearizes the dynamics including the saturated controller defined in Eq.~\eqref{Eq:Control} while the second linearizes the controller without the saturation nonlinearity. In essence, the latter choice assumes the feedback control is not subject to the control limits, and as a result overpredicts the efficacy of the controller, as seen by the large underpredictions of the standard deviation of both altitude and range. Closed-loop robust optimal control was explored using linear sensitivity propagation without considering saturation in \cite{MarsEntryDesensitized}.
Both linearizations once again underpredict the mean altitude substantially. The UT overpredicts the mean range by 900 m and otherwise is once again accurate to within a few hundred meters or less for the remaining statistics. Note that this control profile has significant margin throughout the trajectory except close to the final velocity; in this case control saturation is likely not as significant as for reference control profiles that spend more time at or near the control limits. The UT is expected to produce greater benefits over LC in such scenarios.

\begin{figure}[h!]
	\centering
	\includegraphics[width=1\textwidth]{Images/AlphaSensitivity}
	\caption{Sensitivity of statistics to the unscented transform scaling parameter}
	\label{Fig:AlphaSensitivity}
\end{figure}

Figure~\ref{Fig:AlphaSensitivity} shows how the UT-estimated means and standard deviations vary with different choices of the scaling parameter. Note that the mean range has been shifted down 300 km to fit it in the same figure. The altitude statistics are quite stable and vary only slightly with $\alpha$, while the effect is much stronger for range. This is a general phenomenon we have observed across numerous control profiles; as a result we recommend selection of $\alpha$ based on range considerations. 
Both the mean range and its standard deviation vary monotonically with $\alpha$. In this example, the mean range error is minimized for $\alpha = 20$, while $\sigma_s$ is approximately optimized for $\alpha = 13$. Our choice $\alpha=15$ strikes a balance between these two. 
While the tuning parameter $\alpha$ is important, we note that in this example the UT estimates of $\bar{h},\,\sigma_h$, and $\bar{s}$ are more accurate than either LC method for all of $\alpha$ values considered, and $\sigma_s$ is better for several values.
This example is meant to motivate our choice of the UT over LC; more detailed examples utilizing optimal solutions are presented in Chapter~\ref{Ch:NumericalAssessment}. 
%TODO: Show control profile, trajectory plots with MC samples, UT points? Or maybe just estimated 3-sigma bounds but might be too messy with three on there Also show a table of the terminal statistics. 

%Sensitivity to $\alpha$  
%Talk about the (weak) dependence on alpha, such that we can choose a single decent value and use it for all of the solutions? 

\section{Differential Dynamic Programming}\label{Sec:DDP}
we begin this section by reviewing the control-limited differential dynamic programming method, proposed in Ref.~\cite{DDP_ControlLimited}, that will be used as the basis for solving the robust optimal guidance problem. The following subsections discuss modifications to the algorithm and the problem formulation that will finally enable the solution of the problem. In the first subsection we present a extension of the algorithm that allows for the joint optimization of static design parameters. The following subsection proposes a simplification that vastly reduces computational time and memory requirements at the expense of lower convergence rate. This simplification is instrumental in solving the large-scale problem. Next, we discuss a reformulation of the performance objective into a Lagrange (or running) cost instead of a Mayer (or terminal) cost. Finally, we present a differentiable approximation to the saturation nonlinearity in the controller, Eq.~\eqref{Eq:Control}. Since DDP requires derivatives of the dynamic equations with respect to the reference control, state variables, and potentially the feedback gains, replacing the non-differentiable saturation function is essential to the numerical solution.

This particular DDP algorithm is formulated in discrete time, so the continuous time dynamics must be discretized. In our numerical examples, Euler integration is used:
\begin{align}
	\state_{i+1} = \mathbf{f}(v, \state_i,\control_i) = \state_i + \state_i'\Delta v \label{Eq:DiscreteDynamics}
\end{align}
A trajectory $\{\State,\Control\}$ is a sequence of controls $ \Control=\{\control_0,\control_1,...,\control_{N-1}\} $ and corresponding states $\State=\{\state_0,\state_1,...,\state_N\}$ determined by integrating \eqref{Eq:DiscreteDynamics} from $\state_0$.
Although the optimal control objective as posed considers only a terminal cost, in this section we consider a generic cost function $J$ consisting of a sum of running costs $l$ and a terminal cost $l_N$:
\begin{align}
	J(\state_0,\Control) = \sum_{i=0}^{N-1}l(\state_i,\control_i) + l_N(\state_N)
\end{align}
Let $\Control_i$ be the tail of the control sequence, $\{\control_i,\control_{i+1},...,\control_{N-1}\}$, and the cost-to-go $J_i$, defined as the partial sum of costs from $i$ to $N$ is
\begin{align}
	J_i(\state,\Control_i) = \sum_{j=i}^{N-1}l(\state_j,\control_j) + l_N(\state_N)
\end{align}
The value function at timestep $i$ is the optimal cost-to-go at \state
\begin{align}
	V_i(\state) = \min_{\Control_i} J(\state, \Control_i)
\end{align}
and at the final timestep the value function is equal to the terminal objective. The dynamic programming principle reduces the problem of minimization over $\Control_i$ to a sequence of minimization problems over $u$ at each timestep 
\begin{align}
	V(\state) = \min_{\control}\left[l(\state,\control) + V^+(\mathbf{f}(\state,\control))\right] \label{eq_dynamic_programming}
\end{align}
where $V^+$ is the value at the next time step.
Let the pseudo-Hamiltonian $Q(\delta\state,\delta\control)$ be the change in value function as a function of perturbations to the pair $(\state,\control)$:
\begin{align}
	Q(\delta\state,\delta\control) = l(\state+\delta\state,\control+\delta\control) + V^+(\mathbf{f}(\state+\delta\state,\control+\delta\control))
\end{align}
The second-order expansion of $ Q $ is given by
\begin{align}
	Q_\state &= l_\state + \mathbf{f}_\state^T V^+_\state \\
	Q_\control &= l_\control + \mathbf{f}_\control^T V^+_\state \\
	Q_{\state\state} &= l_{\state\state} + \mathbf{f}_\state^T V^+_{\state\state}\mathbf{f}_\state + V^+_\state \mathbf{f}_{\state\state} \label{eq_hessian1}\\
	Q_{\control\state} &= l_{\control\state} + \mathbf{f}_\control^T V^+_{\state\state}\mathbf{f}_\state + V^+_\state \mathbf{f}_{\control\state} \label{eq_hessian2}\\
	Q_{\control\control} &= l_{\control\control} + \mathbf{f}_\control^T V^+_{\state\state}\mathbf{f}_\control + V^+_\state \mathbf{f}_{\control\control} +\lambda I \label{eq_hessian3}
\end{align}
where the subscripts denote partial derivatives with respect to that quantity, and the final term in each of the Hessian equations (\ref{eq_hessian1})-(\ref{eq_hessian3}) are tensor-vector contractions, and $ \lambda $ is a regularization parameter.  %TODO: rewrite them as sums over the 'pages' of the hessian

The optimal control modification $\delta\control^*$ for some perturbation $\delta\state$ is a locally-linear feedback control $\delta\control^* = \mathbf{k} + K\delta\state$ obtained by minimizing the quadratic model subject to linear bounds on the controls
\begin{align}
	\mathbf{k} = &\arg\min_{\delta\control} Q(\delta\state,\delta\control) \\
	&\mathrm{subject\,to\,\;} \nonumber\\
	\control_{\min}\le &\control+\delta\control \le\control_{\max}
\end{align}
and $K = -Q_{\control\control}^{-1}Q_{\control\state}$. Substituting the optimal control into the expansion of $Q$, a quadratic model of $V$ is obtained with derivatives
\begin{align}
	\begin{split}
		\label{eq_value_recurse}
		V_\state &= Q_{\state}- K^TQ_{\control\control}\mathbf{k}\\
		V_{\state\state} &= Q_{\state\state} - K^TQ_{\control\control}K.
	\end{split}
\end{align}
The backward pass is performed by initializing $V$ and its derivatives with the value and derivatives of the terminal objective, then recursively computing the optimal control policy and Eq.~(\ref{eq_value_recurse}). 

The forward pass uses the newly computed control policy to integrate the new trajectory, subject to a backtracking line search
\begin{align}
	\hat{\state}_0 &= \state_0 \\
	\hat{\control}_{i} &= \control_i + \epsilon \mathbf{k}_i + K_i(\hat{\state}_i - \state_i)\\
	\hat{\state}_0 &= f(\hat{\state}_i,\hat{\control}_i)
\end{align}
where $\epsilon$ is a search parameter initialized to 1 and reduced until the value function shows improvement. The backward-forward iterations are repeated until convergence to a locally optimal trajectory. In the event the backward pass fails to produce a descent direction, the regularization parameter $\lambda$ is increased and the backward pass is repeated. See Ref.~\cite{DDP_ControlLimited} for more details about the line search and regularization procedures. 

%TODO: maybe move this into the DDP section and discussion from the start? Call it a modification to treat the EFPA, that can also be used for other constant design parameters by appending them to the state? Hard to justift that approach when Im not optimizing the EFPA though 
\subsection{Optimal Design Parameters}\label{Sec:DesignOptimization}
Previously, we considered the state vector to be a function of the independent variable and the uncertainty vector $\param$. However, we may wish to treat the state as a function of some additional constant design parameters, $\design$, $\state(v, \param, \design)$. The design parameters are different from the uncertain parameters because they do not increase the number of sigma points required. Additionally, in our formulation, the design parameters are appended only to the extended vector of all sigma points. Thus, given state dimension $n_x$, uncertainty dimension $n_p$, and design parameter dimension $n_d$, the total number of states is $n_x(2n_p+1) + n_d$. We define an augmented state vector $\state_a = [\xut, \design]^T$ 
and augment the entry dynamics with trivial dynamics
\begin{align}
	\state_a' = \left[
	\begin{matrix}
		\xut' \\
		\mathbf{0}
	\end{matrix}
	\right]
\end{align}

DDP is now applied to the augmented state vector with the following modification. 
During the backward pass, the first and second partial derivatives of the value function are computed recursively from the final velocity to the initial velocity.
At the end of the backward pass, we generally set $\delta\state_a=\zero$ because we are optimizing a trajectory from a fixed initial state. However, when including design parameters that can be altered, we instead derive a relationship to update them to reduce the value function based on the value function derivatives, i.e. $\delta\state_a = [\zero,\delta\design]^T$

Define the block matrix, and note that we invert a matrix in $\mathbb{R}^{n_d\times n_d}$. Eigendecompose, and replace small/negative eigenvalues with regularization term - shifting from Newton-like to gradient descent 
The gradient and hessian of the value function can be written in block notation as 
\begin{align}
	V_{\state_a} &= \begin{bmatrix}
		V_\state & V_\design
	\end{bmatrix} \\
	V_{\state_a\state_a} &= \begin{bmatrix}
	V_{\state\state} & V_{\state\design} \\
	V_{\design\state} & V_{\design\design}
\end{bmatrix}
\end{align}

This can be used to jointly optimize the feedback gains, entry flight path angle and more. This is motivated by the ``overfitting" problem. 

\subsection{Second Order Dynamics Derivatives}\label{Sec:DDP_Simplification}
%TODO: Show an example with and without the second order terms to compare convergence. Either here, or maybe in the numerical assessment section. 
The second derivatives $\mathbf{f}_{\state\state},\, \mathbf{f}_{\control\state}$ in Eqs.\eqref{eq_hessian1}) and \eqref{eq_hessian2} are $n\times n\times n$ and $m\times n\times n$ tensors, respectively, that must either be stored at $N$ timesteps, or else recomputed in the event the backward pass must be repeated with increased regularization. For large $n$, both computing and storing these quantities is problematic. Unfortunately, removing these terms entirely, as is done in the iterative Linear Quadratic Gaussian method \cite{iLQG}, leads to poor performance for the problem under consideration. In particular, the first-order algorithm has a noticeably smaller radius of convergence, and different guesses at the initial control sequence often lead to different local minima. In contrast, the full second-order DDP algorithm consistently converges to a single minimum from most initial guesses. Thus, we propose a compromise in which only the $n\times m \times m$ tensor $f_{\control\control}$ is retained in the above equations. Numerical experiments suggest this is sufficient to avoid poor solutions, at the expense of a slower convergence rate than the full algorithm with all second-order terms included. However, in the situation at hand, in which the number of controls $m$ is fixed and small ($ m=1 $), and the augmented state dimension, which depends on the number of uncertainties under consideration, is large ($n=78$ for $\param\in\mathbb{R}^6$), this modification proves to be enabling for $N\geq 100$. A limited-memory version of the Quasi-Newton approximations to these terms proposed in \cite{QNDDP} may also be a good choice to obtain superlinear convergence properties of the approximated second-order algorithm while reducing the memory required. 
%(This reduces complexity from cubic in $n$ to linear)

\subsection{Objective Reformulation}
In order for DDP to be applied to our robust optimal control problem, it is essential to reformulate some or all of the objective terms as running costs rather than terminal costs. This is because the terminal formulation results in singularity of the Hessian of the pseudo-Hamiltonian with respect to the control, and this quantity must be invertible for the algorithm to proceed. 
%In practice, one could also add a small control regularization term to the objective, but experience indicates that reformulating the mean altitude objective is preferable. 
The reformulation is based on the relationship 
\begin{align}
	J(\state(v_f)) = J(\state_0) + \int_{v_0}^{v_f}J'\mathrm{d}v \label{Eq:GenericCostRate}
\end{align}
For a fixed initial state, the term $J(\state_0)$ is constant and cannot be affected by the control profile. As such, 
\begin{align}
	u^* &= \arg\min \,J(\state(v_f)) \\
	&= \arg\min\int_{v_0}^{v_f}J'\mathrm{d}v
\end{align}
Thus, we need an expression for $J'$ for the guidance objective defined in Eq.~\eqref{Eq:Objective}. Due to linearity of the expectation operator, we have the following relationships for the rate of change of the first two moments of a scalar random variable
%\begin{align}
%\frac{d }{d t}\E{x} &= \E{\dot{x}} \label{Eq:MeanRate}\\
%\frac{d }{d t}\V{x} &= \E{2x\dot{x}} - 2\E{x}\E{\dot{x}} \\
%\sigma_x &= \V{x}^{\frac{1}{2}} \\
%\frac{d }{d t}\sigma_x &= \frac{1}{2\sigma_x}\frac{d }{d t}\V{x} \\
% &= \frac{1}{\sigma_x}\left(\E{x\dot{x}} - \E{x}\E{\dot{x}}\right) \label{Eq:StdRate}
%\end{align}
%Equations~\eqref{Eq:MeanRate} and \eqref{Eq:StdRate} are converted into velocity rates
\begin{align}
	\frac{d }{d v}\E{x} &= \E{x'} \label{Eq:MeanRateVel}\\
	\frac{d }{d v}\V{x} &= \E{2xx'} - 2\E{x}\E{x'} \\
	\sigma_x &= \V{x}^{\frac{1}{2}} \\
	\frac{d }{d v}\sigma_x &= \frac{1}{\sigma_x}\left(\E{xx'} - \E{x}\E{x'}\right) \label{Eq:StdRateVel}
	%\frac{d }{d t}\V{x}^{\frac{1}{2}} &= \frac{1}{2\V{x}^{\frac{1}{2}}}\frac{d }{d t}\V{x}
\end{align}
Applying these to altitude and downrange distance, and discretizing the resulting integral, we arrive at the 
the reformulated discrete-time running cost
\begin{align}
	%J = \int_{v_0}^{v_f}-\E{h'} +  2w_h(\E{hh'}-\E{h}\E{h'}) + 2w_s(\E{ss'}-\E{s}\E{s'})\mathrm{d}v \\
	%J = \sum_{i=0}^{N-1} \left[-\E{h'} +  2w_h(\E{hh'}-\E{h}\E{h'}) + 2w_s(\E{ss'}-\E{s}\E{s'})\right]\delta v 
	l(\state,\control) = \left(-\E{h'} +  \frac{w_h}{\sigma_h}(\E{hh'}-\E{h}\E{h'}) + \frac{w_s}{\sigma_s}(\E{ss'}-\E{s}\E{s'})\right)\Delta v
\end{align}
While reformulating the mean altitude objective is essential to solving the problem, especially when $w_s=w_h=0$, reformulating the standard deviation terms is not. However, in numerical experiments we found that while both formulations result in the same solution, using running costs for the standard deviations allows DDP to converge in noticeably fewer iteration. Running costs are used in the results in this dissertation.

\subsection{Saturation}
The saturation function in Eq.~\eqref{Eq:Control} is not differentiable at the  saturation limits, which produces numerical difficulties for the DDP algorithm. To circumvent this problem, we substitute the following smooth approximation when solving the robust optimal guidance problem
\begin{align*}
	\mathrm{sat}_{[0,1]}(x) \approx \frac{1}{2} + \frac{1}{4M}\log\left(\frac{\cosh (2Mx)}{\cosh (2M(x-1))}\right) 
\end{align*}
where $M>0$ is a tuning parameter. Larger values of $M$ reduce approximation error but increase the derivative magnitude at the saturation limits. $M=20$ reduces the maximum error to less than $1\%$ and is used in our implementation. Figure~\ref{Fig:SmoothSat} shows how the approximation varies with $M$; at $M=20$, the approximation is visually indistinguishable from the true function.
\begin{figure}[h!]
	\centering
	\includegraphics[width=1\textwidth]{Images/SmoothSat}
	\caption{The saturation function and smooth approximation for several values of the tuning parameter.}
	\label{Fig:SmoothSat}
\end{figure}



%\section{UT-Approximated Robust Guidance Problem}
%Let \xut = 
%\begin{align}
%	&\min J = -\bar{h}(v_f) + w_h\sigma_h(v_f) + w_s\sigma_s(v_f) \nonumber\\
%	&\quad\quad\qquad\mathrm{subject\, to }\nonumber \\
%	&\dot{\xl}(v,\param) = \dynamics_v(v, \xl(v,\param), u(v), \param),\quad
%	\xl(v_0,\param) = \state_0(\param) \nonumber\\
%	&\param\sim \normal(\mathbf{0},\cov_{\param}) \nonumber\\
%	&0 \le \ur(v) \le 1 \quad \forall\,v\in [v_0, v_f] \nonumber
%\end{align}

%\section{Necessary Conditions}
%In this section the necessary conditions for an optimal solution to the robust optimal guidance problem are given. Since the terminal objective function has been reformulated as a Lagrange (or running) cost
%\begin{align}
%	J = \int_{v_0}^{v_f}\left[ -\E{h'} +  \frac{w_h}{\sigma_h}(\E{hh'}-\E{h}\E{h'}) + \frac{w_s}{\sigma_s}(\E{ss'}-\E{s}\E{s'})\right]\, \mathrm{d}v
%\end{align}
%and thus the Lagrange cost is
%\begin{align}
%	L(x,u) =  -\E{h'} +  \frac{w_h}{\sigma_h}(\E{hh'}-\E{h}\E{h'}) + \frac{w_s}{\sigma_s}(\E{ss'}-\E{s}\E{s'})
%\end{align}
%The UT approximation of $L$ is given by
%\begin{align}
%	L_{UT}(x,u) =  -\EUT{h'} +  \frac{w_h}{\sigma_h}(\EUT{hh'}-\EUT{h}\EUT{h'}) + \frac{w_s}{\sigma_s}(\EUT{ss'}-\EUT{s}\EUT{s'})
%\end{align}
%
%The Hamiltonian for the UT-approximated ROCP is 
%$H = \lambda^Tf(v,\state(v),\param)+ L(\state,\param)$ 


%%% Local Variables: ***
%%% mode: latex ***
%%% TeX-master: "thesis.tex" ***
%%% End: ***

\chapter{Guidance Assessment Conditions}\label{Ch:AssessmentConditions}

%Also talk about algorithm parameters used to generate trajectories
\section{Entry Conditions and Targets}
State and dispersions at the entry interface. State and dispersions at the range control initiation velocity 
Mass = 2804 kg, Area = 15.8 $\mathrm{m}^2$


\section{Robust Optimal Control Solution Parameters}
Linesearch rules? M = 20 for smooth saturation. N=250 velocity steps, with four Euler integration steps per velocity step. 

%Guess trajectory, show it and discuss. 
In all numerical examples, the initial guess at the optimal reference control is a linear ramp. This guess is motivated by experience with altitude optimal trajectories, which feature an initial zero lift arc followed by a full lift up arc. Such a control has zero margin (in one direction) at every velocity. Our choice is motivated by a similar lift down to lift up change, but more gradual and with significant margin throughout. 

All solutions consider static gains. In the first section we consider a single set of gains $=[]$ to highlight the impact of the reference trajectory. Later results use jointly optimized gains. For these cases, the initial guess at the optimal gains is the same set.

\subsection{Selecting the UT Parameter}
Show results for a fixed profile sweeping alpha and show that the estimates are monotonic wrt to the parameter. As result, there exists an alpha that minimizes the error in altitude and range, generally not the same alpha. 

Our earlier example showed that selecting $\alpha$ is important, especially for the range variable. In the numerical assessment, we consider vary weight combinations on the grid $w_h\times w_s = [0,3]\times[0,3]$. As such, we selected $\alpha$ to minimize the UT-estimation error in $\sigma_s$ for the optimal solution with $w_h=w_s=1.5$, i.e., at the center point of the grid. 

\section{Simulation}
Monte Carlo setup - integration wrt time, 1Hz integration rate, 

%Primary metrics include the 3$\sigma$ low altitude and the 3$\sigma$ range error at the terminal velocity. Some discussion of the mean altitude and range as well. 
Bank rate is limited to $20^{\circ}/s$
Monte Carlo simulation with 3000 sample trajectories is conducted. The samples are drawn using Latin Hypercube sampling. The bank angle command is updated at a rate of 1 Hz. Range error is the distance between the terminal downrange distance defined by the reference trajectory and the downrange distance flown. The primary metrics we will examine are the 3$\sigma$-low altitude $=\bar{h}-3\sigma_h$ and 3$\sigma$ range error $= 3\sigma_s$ at the final velocity. The terminal state distributions are generally non-Gaussian, and in practice percentiles are used to specify mission requirements. However, metrics based on the mean and standard deviation are presented, because these are the values estimated by the unscented transform, which allows for a direct comparison with the Monte Carlo results.

%%% Local Variables: ***
%%% mode: latex ***
%%% TeX-master: "thesis.tex" ***
%%% End: ***

\chapter{Guidance Assessment}\label{Ch:NumericalAssessment}

\section{Reference Trajectory Optimization}
In this section we solve the ROGP for a variety of objective weights $(w_h,\,w_s)$ while holding the gains constant. This demonstrates the extent to which the reference trajectory can impact the solution for a fixed set of gains. In the first subsection, we consider the open-loop ($K=\zero$) scenario to gauge the level of dispersions that must be mitigated by the entry guidance, and the extent to which open-loop covariance shaping can help. MSL entry guidance designers performed a similar exercise~\cite{MSL_EDL2}. The second subsection repeats the sweep over the weights in a closed-loop scenario.

\subsection{Open-Loop Optimization}
Figure~\ref{Fig:MCResultsOpenLoop} presents contours of the Monte Carlo estimated 3$\sigma$-low altitude and range errors. The 3$\sigma$ range errors vary from 59 to 66 km, while the 3$\sigma$-low altitude varies from 7.8 km to 10.6 km. Increasing the weight $w_s$ reduces range errors as intended but comes at a penalty to altitude performance. The results also highlight the magnitude of the terminal state variations arising from the variations in the initial state and model uncertainty. 
%TODO: Show some control profiles? 
\begin{figure}[h!]
	\centering
	\includegraphics[width=1\textwidth]{Images/OpenLoop_WeightSweepMCResults}
	\caption{Monte Carlo statistics for the open-loop trajectory optimization. }
	\label{Fig:MCResultsOpenLoop}
\end{figure}
%\begin{figure}[h!]
%	\centering
%	\includegraphics[width=1\textwidth]{Images/Reoptimized_WeightSweepError}
%	\caption{}
%	\label{Fig:MCErrorsOpenLoop}
%\end{figure}

\subsection{Closed-Loop Optimization}
%Maybe show ONLY  Monte Carlo results here. Then in the next section, do a detailed comparison. Justified by the fact that optimized gains are what we would want to use, so we should confirm those? But MC here already confirms. We could maybe get away with showing the UT only
Figure~\ref{Fig:MCResultsFixedGain} presents contours of the Monte Carlo estimated 3$\sigma$-low altitude and range errors for the closed-loop guidance with $K=K_2$. Introducing feedback allows the guidance to achieve 3$\sigma$ range errors as low as 5.6 km while also keeping the 3$\sigma$-low altitude above 9 km. Figure~\ref{Fig:MCErrorsFixedGain} shows the contours of the errors in the UT statistics compared to the Monte Carlo results. The error in 3$\sigma$-low altitude varies between 120 m overestimate and a 360 underestimate. For a majority of $(w_h,w_s)$ pairs, the low altitude is underestimated. 
\begin{figure}[h!]
	\centering
	\includegraphics[width=1\textwidth]{Images/Reestimated_WeightSweepMCResults}
	\caption{Monte Carlo statistics for the closed-loop trajectory optimization with static gains.}
	\label{Fig:MCResultsFixedGain}
\end{figure}
\begin{figure}[h!]
	\centering
	\includegraphics[width=1\textwidth]{Images/Reestimated_WeightSweepError}
	\caption{Unscented Transform estimation errors relative to Monte Carlo statistics for trajectory optimization with static feedback gains.}
	\label{Fig:MCErrorsFixedGain}
\end{figure}


\section{Joint Gain Optimization}
%Motivate the constant gain optimization here by showing an example of bad MC performance when simultaneously optimizing velocity-varying gains? Then show how well the constant gains can do when jointly optimized. 

In this section, the ROGP is again solved for a variety of weights but now the gains are treated as design parameters, and the DDP modification presented in Subsection~\ref{Sec:DesignOptimization} is used to jointly optimize them alongside the reference trajectory and control. As one might intuitively expect, the effect is much more extreme with joint optimization. 

An additional 800 meters of low altitude can be achieved (relative to the previous unoptimized gains), but at the expense of huge range errors. However, a very tight range distribution of $3\sigma_s = 3$ km can be achieved while still maintaining a 9 km 3$\sigma$-low altitude.  
\begin{figure}[h!]
	\centering
	\includegraphics[width=1\textwidth]{Images/Reoptimized_WeightSweepMCResults}
	\caption{Monte Carlo statistics for jointly optimized static feedback gains.}
	\label{Fig:MCResultsOptGain}
\end{figure}
\begin{figure}[h!]
	\centering
	\includegraphics[width=1\textwidth]{Images/Reoptimized_WeightSweepError}
	\caption{Unscented Transform estimation errors relative to Monte Carlo statistics for jointly optimized static feedback gains.}
	\label{Fig:MCErrorsOptGain}
\end{figure}
%TODO: plot mean (=target) range to determine if longer or shorter trajectories are more robust. 

\section{Detailed Comparison}
Other sections are input-output, examining the tradeoffs enabled by varying the weights in the performance index. In this section we present an in-depth look at a single solution, including visualizing the samples over time. We also perform further analysis to determine which uncertainties most strongly affect performance by applying the global sensitivity analysis technique known as Monte Carlo Filtering \cite{MonteCarloFiltering}. 

%TODO


\section{Convergence}
In this section we examine the convergence of solutions using the full DDP algorithm, with the $\mathbf{f}_{\state\state}=\zero,\, \mathbf{f}_{\control\state}=\zero$ modification proposed in Subsection~\ref{Sec:DDP_Simplification}, and the iterated Linear Quadratic Regulator (iLQR) variation where additionally $\mathbf{f}_{\control\control}=\zero$. We solve the ROGP for $w_h=w_s=1$ with each method. The primary metrics that we are interested in are the total cost, the total number of iterations, the time spent computing derivatives, and the maximum of the norm of the gradient of the value function with respect to the controls over the entire trajectory. We will also examine the linesearch stepsizes and regularization parameters.
%Have to remove either state or parameter uncertainty to be able to use the full derivatives formulation.

\begin{figure}[h!]
	\centering
	\includegraphics[width=1\textwidth]{Images/Convergence/ControlProfiles}
	\caption{Total cost per iteration}
	\label{Fig:ConvergeControls}
\end{figure}
\begin{figure}[h!]
	\centering
	\includegraphics[width=1\textwidth]{Images/Convergence/cost}
	\caption{Total cost per iteration}
	\label{Fig:ConvergeCost}
\end{figure}
\begin{figure}[h!]
	\centering
	\includegraphics[width=1\textwidth]{Images/Convergence/grad_norm}
	\caption{Maximum control gradient norm over all timesteps.}
	\label{Fig:ConvergeGradient}
\end{figure}
\begin{figure}[h!]
	\centering
	\includegraphics[width=1\textwidth]{Images/Convergence/alpha}
	\caption{Backtracking linesearch stepsize per iteration.}
	\label{Fig:ConvergeStepsize}
\end{figure}
\begin{figure}[h!]
	\centering
	\includegraphics[width=1\textwidth]{Images/Convergence/time_derivs}
	\caption{Time in seconds spent computing derivatives of the cost and dynamics at each iteration.}
	\label{Fig:ConvergeTime}
\end{figure}
%TODO: SHow control profiles? All three converge to similar but not exactly the same profile. 
Figure~\ref{Fig:ConvergeStepsize} shows the total cost at each iteration. Full DDP achieves the lowest cost of $2.864$ after 81 iterations. Control Hessian-only DDP achieves a cost of 2.870 after 46 iterations, and iLQR has the highest cost, 2.871, and still has not converged after 250 iterations. Figure~\ref{Fig:ConvergeGradient} shows the maximum gradient at each iteration. The iLQR solution does not appear to reach a stationary point. 
Figure~\ref{Fig:ConvergeStepsize} shows the stepsizes accepted in the linesearch procedure. Both DDP variants frequently accepted the full step $\epsilon=1$, while iLQR always took considerably smaller steps on the order of $ 10^{-2} $. Finally, Fig.~\ref{Fig:ConvergeTime} shows the drastic, nearly 10x reduction in time spent computing the derivatives resulting from omitting Hessian terms. The average times were $47.7$ s, 5.6 s, and 5.2 s, respectively. 
This example shows that retaining the second order derivatives with respect to the controls achieves nearly the same convergence as the full DDP algorithm while being only mildly more expensive to compute than the iLQR solution. 

%TODO: sensitivity to initial guess? 

%\section{Joint Optimization of Velocity-Varying Gains}
%Show negative result and describe overfitting problem. 

%\section{As Vehicle Design Tool}
%What L/D is required to achieve similar entry altitudes for a heavier vehicle of $\beta=200$? We use a simple scaling of the L/D profile $L/D$. The gains are jointly optimized, and additionally, the entry flight path angle is treated as a design variable. To estimate a useful lower bound on the L/D, we select $w_h=3, w_s=0$; the desired solution will likely place non-zero weight on the range errors, and thus need even more L/D. 
%%% Local Variables: ***
%%% mode: latex ***
%%% TeX-master: "thesis.tex" ***
%%% End: ***

\chapter{Conclusions}

In linear systems subject to Gaussian uncertainties, the covariance matrix evolves independently of the mean trajectory, and must be shaped by feedback control. The natural nonlinearities of the entry problem, particularly the aerodynamic accelerations, which are quadratic in velocity and approximately exponential in altitude, result in drastically different evolution of the state covariance matrix along different trajectories through the atmosphere. This allows for the possibility of open-loop covariance shaping. Introducing a feedback control allows for even greater shaping of the state covariance as well as an additional nonlinearity - saturation. 

\section{Future Work}
Lateral guidance, and decision to change to third phase guidance, such as heading alignment or deployment position alignment \cite{GuangfeiDissertation}.
Navigational problem? Combined optimal estimation and control, also inclusion of nav error in guidance. 

Constrained optimization to make the EPFA optimization more useful, Gload/heating etc. 

%%% Local Variables: ***
%%% mode: latex ***
%%% TeX-master: "thesis.tex" ***
%%% End: ***

% ... and so on

% These commands fix an odd problem in which the bibliography line
% of the Table of Contents shows the wrong page number.
\clearpage
\phantomsection

% "References should be formatted in style most common in discipline",
% abbrv is only a suggestion.
\bibliographystyle{abbrv}
\bibliography{bib}

% The Thesis Manual says not to include appendix figures and tables in
% the List of Figures and Tables, respectively, so these commands from
% the caption package turn it off from this point onwards. If needed,
% it can be re-enabled later (using list=yes argument).
\captionsetup[figure]{list=no}
\captionsetup[table]{list=no}

% If you have an appendix, it should come after the references.
%\begin{appendices}
%\include{appendix}
%\end{appendices}

\end{document}