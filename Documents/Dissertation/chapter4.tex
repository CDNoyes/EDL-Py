\chapter{Guidance Strategy}
At its core, the guidance strategy for longitudinal range control is extremely simple, consisting of an open-loop vertical $L/D$ and a feedback law, subject to physical- and safety-related constraints. Thus, to design the guidance means to design the reference vertical $L/D$ and corresponding reference trajectory, and the feedback gains. 

The physical constraint is that the available lift is limited, so the most vertical lift that can be achieved is with zero bank angle, or $u=\cos\sigma=1$. The lower bound is dependent on the mission and vehicle characteristics. Physics again limits the magnitude to $u=-1$, but it is often prudent to further limit the lower bound due to the low $L/D$ of typical Mars entry vehicles.

\section{}
\subsection{Objective Function}
The first objective is to maximize the mean altitude at the final velocity while minimizing the altitude standard deviation
\begin{equation}
	\max J_h = \bar{h}(v_f) - w_h\sigma_h(v_f) \label{Eq:AltitudeObjective}
\end{equation}
where $w_h\ge0$ is a penalty on the standard deviation. Maximizing Eq.~(\ref{Eq:AltitudeObjective}) for $w_h$ results in an optimal mean altitude, while 

The second objective is to minimize the standard deviation of trajectory length, consistent with the range control objective, is 
\begin{equation}
	\min J_s = w_s\sigma_s(v_f) \label{Eq:RangeObjective}
\end{equation}

\section{}

%%% Local Variables: ***
%%% mode: latex ***
%%% TeX-master: "thesis.tex" ***
%%% End: ***
