\thesistitle{Robust Optimal Entry Guidance for High Elevation Mars Landing}

%"Dissertation" for PhD, "Thesis" for master's
\documenttitle{Dissertation}

\degreename{Doctor of Philosophy}

% Use the wording given in the official list of degrees awarded by UCI:
% http://www.rgs.uci.edu/grad/academic/degrees_offered.htm
\degreefield{Mechanical and Aerospace Engineering}

% Your name as it appears on official UCI records.
\authorname{Connor David Noyes}

% Use the full name of each committee member and full title 
% (e.g. Professor/Associate Professor).
\committeechair{Professor Kenneth D. Mease}
\othercommitteemembers
{
  Professor Athanasios Sideris\\
  Professor Tryphon Georgiou
}

\degreeyear{2021}

\copyrightdeclaration
{
  {\copyright} {\Degreeyear} \Authorname
}

% If you have previously published parts of your manuscript, you must list the
% copyright holders; see Section 3.2 of the UCI Thesis and Dissertation Manual.
% Otherwise, this section may be omitted.
% \prepublishedcopyrightdeclaration
% {
% 	Chapter 4 {\copyright} 2003 Springer-Verlag \\
% 	Portion of Chapter 5 {\copyright} 1999 John Wiley \& Sons, Inc. \\
% 	All other materials {\copyright} {\Degreeyear} \Authorname
% }

% The dedication page is optional
% (comment out to exclude).
\dedications
{ 
  To my father Michael and my mother Diane.
}

\acknowledgments
{
  I would first like to thank my advisor Kenneth D. Mease, whose expertise shaped not only the course of my research, but also my entire career. 
  
  I would like to thank my fellow UCI colleagues, especially Drs. Guangfei Duan, Eric Trumbauer, and Alessandro Bombelli, for many fruitful discussions during our time together.
  
  I am also indebted to my committee members, Profs. Sideris and Georgiou, for their expedited review and suggestions. 
  
  I would also like to thank my colleagues at the Jet Propulsion Laboratory for their advice, and for their patience. My experience at JPL profoundly shaped the direction of my research culminating in this dissertation. Special thanks to Joel Benito, whose guidance and mentorship taught me so much in so little time, and whose friendship I am continually grateful for. I am a better engineer for having known you. 
  
  I gratefully acknowledge the Holmes Endowed Fellowship for it financial support. 
  
  Finally, I am immensely grateful to my mother, whose constant support allowed me to pursue this degree.
  
%  You also need to acknowledge any publishers of your previous
%  work who have given you permission to incorporate that work
%  into your dissertation. See Section 3.2 of the UCI Thesis and
%  Dissertation Manual.)
}


% Some custom commands for your list of publications and software.
\newcommand{\mypubentry}[3]{
  \begin{tabular*}{1\textwidth}{@{\extracolsep{\fill}}p{4.5in}r}
    \textbf{#1} & \textbf{#2} \\ 
    \multicolumn{2}{@{\extracolsep{\fill}}p{.95\textwidth}}{#3}\vspace{6pt} \\
  \end{tabular*}
}
\newcommand{\mysoftentry}[3]{
  \begin{tabular*}{1\textwidth}{@{\extracolsep{\fill}}lr}
    \textbf{#1} & \url{#2} \\
    \multicolumn{2}{@{\extracolsep{\fill}}p{.95\textwidth}}
    {\emph{#3}}\vspace{-6pt} \\
  \end{tabular*}
}

% Include, at minimum, a listing of your degrees and educational
% achievements with dates and the school where the degrees were
% earned. This should include the degree currently being
% attained. Other than that it's mostly up to you what to include here
% and how to format it, below is just an example.
%
% CV is required for PhD theses, but not Master's
% comment out to exclude
\curriculumvitae
{

\textbf{EDUCATION}
  
  \begin{tabular*}{1\textwidth}{@{\extracolsep{\fill}}lr}
    \textbf{Doctor of Philosophy in Aerospace Engineering} & \textbf{2021} \\
    \vspace{6pt}
    University of California & \emph{Irvine, CA} \\
    \textbf{Master of Science in Aerospace Engineering} & \textbf{2013} \\
    \vspace{6pt}
    California Polytechnic State University & \emph{San Luis Obispo, CA} \\
    \textbf{Bachelor of Science in Aerospace Engineering} & \textbf{2014} \\
    \vspace{6pt}
    California Polytechnic State University & \emph{San Luis Obispo, CA} \\
  \end{tabular*}

\vspace{12pt}
\textbf{ENGINEERING EXPERIENCE}

  \begin{tabular*}{1\textwidth}{@{\extracolsep{\fill}}lr}
    \textbf{Graduate Intern} & \textbf{2015} \\
    \vspace{6pt}
    Jet Propulsion Laboratory & \emph{Pasadena, California} \\
  \end{tabular*}
  
  \begin{tabular*}{1\textwidth}{@{\extracolsep{\fill}}lr}
      \textbf{Guidance and Control Engineer} & \textbf{2015-present} \\
      \vspace{6pt}
      Jet Propulsion Laboratory & \emph{Pasadena, California} \\
    \end{tabular*}

\vspace{12pt}
\textbf{RESEARCH EXPERIENCE}

  \begin{tabular*}{1\textwidth}{@{\extracolsep{\fill}}lr}
    \textbf{Graduate Research Assistant} & \textbf{2013--2021} \\
    \vspace{6pt}
    University of California, Irvine & \emph{Irvine, California} \\
  \end{tabular*}

\vspace{12pt}
\textbf{TEACHING EXPERIENCE}

  \begin{tabular*}{1\textwidth}{@{\extracolsep{\fill}}lr}
    \textbf{Teaching Assistant} & \textbf{2013} \\
    \vspace{6pt}
    California Polytechnic State University & \emph{San Luis Obispo, CA} \\
  \end{tabular*}
  \begin{tabular*}{1\textwidth}{@{\extracolsep{\fill}}lr}
    \textbf{Teaching Assistant} & \textbf{2015} \\
    \vspace{6pt}
    University of California & \emph{Irvine, CA} \\
  \end{tabular*}
\pagebreak

\textbf{PUBLICATIONS}

  \mypubentry{Mars Entry Guidance for High Elevation via Robust Optimal Control}{2021 (Submitted for Review)}{Journal of Spacecraft and Rockets}

%\vspace{12pt}
%\textbf{REFEREED CONFERENCE PUBLICATIONS}

  \mypubentry{Entry Guidance for Propellant Optimal Powered Descent on Mars}{Aug 2020}{AAS/AIAA Astrodynamics Specialist Conference}
  \mypubentry{A Convex Optimization Approach to Mars Entry Trajectory Updating}{Aug 2018}{AAS/AIAA Astrodynamics Specialist Conference}
  \mypubentry{High Ballistic Coefficient Mars EDL With Supersonic Retropropulsion}{Feb 2017}{AAS Guidance and Control Conference}
  \mypubentry{Sensitivity Analysis and Uncertainty Quantification of a Mars Ascent Vehicle Concept}{Aug 2017}{ASME Verification and Validation Symposium}  
%  \mypubentry{Hybrid propulsion Mars Ascent Vehicle concept flight performance analysis}{Sep 2017}{IEEE Aerospace Conference}
%  \mypubentry{Mars Ascent Vehicle Model Simulation}{Aug 2016}{AIAA/AAS Astrodynamics Specialist Conference}


}

% The abstract was previously limited to a maximum of 350 words, 
% but the UCI manual at https://etd.lib.uci.edu/electronic/td2e#2.2.1.
% currently does not indicate that there is any word limit for the abstract
\thesisabstract
{
  Mars landings at higher elevations than achieved to date are desired for scientific pursuits. The phases of atmospheric flight are entry, descent, and landing. The research reported here concerns the guidance for the entry phase. To support higher elevation landing, entry guidance must deliver the entry vehicle to the required altitude with the required horizontal accuracy at the end of the entry phase. The state-of-the-practice entry guidance cannot both raise the final altitude and achieve the required horizontal accuracy at the end of the entry phase. By formalizing the entry guidance objectives as a robust optimal control problem, we seek both to increase the final altitude and to improve the horizontal accuracy. In this dissertation, we consider only the longitudinal motion and investigate the feasibility of determining a reference trajectory that, in closed-loop reference-trajectory-based guidance, will yield the robust performance required for higher elevation landing. To address robustness, the state variables and uncertain parameters in the entry dynamics are treated as random variables using the unscented transformation to approximate their means and variances and state the performance index in terms of these statistics. Differential dynamic programming is used to solve the robust optimal control problem. Case studies of two different classes of entry vehicle demonstrate both the robust performance of the longitudinal entry guidance and the computational feasibility of the design method.

}


%%% Local Variables: ***
%%% mode: latex ***
%%% TeX-master: "thesis.tex" ***
%%% End: ***
