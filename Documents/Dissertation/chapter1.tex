\chapter{Introduction}

This is an example using the \LaTeX{} template for UCI theses and
dissertation documents \cite{uci-thesis-latex}. Figure
\ref{fig:sourcecode} is just for illustration purposes, as is Table
\ref{tab:coordinates}.

\begin{figure}
\begin{verbatim}
#include <iostream>
int main(int argc, char** argv) {
  std::cout << "Hello World." << std::endl;
  return 0;
}
\end{verbatim}
  \caption{Example source code.}
  \label{fig:sourcecode}
\end{figure}

\section{Background}

Lorem ipsum dolor sit amet, consectetur adipisicing elit, sed do
eiusmod tempor incididunt ut labore et dolore magna aliqua. Ut enim ad
minim veniam, quis nostrud exercitation ullamco laboris nisi ut
aliquip ex ea commodo consequat. Duis aute irure dolor in
reprehenderit in voluptate velit esse cillum dolore eu fugiat nulla
pariatur. Excepteur sint occaecat cupidatat non proident, sunt in
culpa qui officia deserunt mollit anim id est laborum.

\begin{table}
  \centering
  \begin{tabular}{|rr|r|}
    \hline
    $x$ & $y$ & $z$ \\
    \hline
    14 & 12 & -2 \\
    0 & 33 & -25 \\
    -3 & 11 & 22 \\
    4 & 4 & 6 \\
    \hline
  \end{tabular}
  \caption{Example coordinates.}
  \label{tab:coordinates}
\end{table}

This dissertation is organized as follows: 
chapter two discusses the modeling used to describe entry trajectories
chapter three describes the entry guidance problem, especially terminal condition(s)
chapter four outlines the guidance strategy, including posing the problem as a ROCP. Possible gain optimization as well. 
chapter five gives an algorithm based on DDP/SLQ to solve the ROCP
chapter six presents the assessment conditions (simulation details, Monte Carlo)
chapter seven presents an assessment of the guidance algorithm 
chapter eight concludes the dissertation


%%% Local Variables: ***
%%% mode: latex ***
%%% TeX-master: "thesis.tex" ***
%%% End: ***
