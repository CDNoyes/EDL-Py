\chapter{Entry Guidance Problem}
We consider the entry guidance problem is to balance the vertical lift of the vehicle to steer it to the target location at the terminal velocity and with sufficient altitude for subsequent landing operations. 

Throughout this dissertation we assume the terminal velocity associated with a parachute based landing, corresponding to deployment at Mach 2. However, this is not the only possibility. We can, for example, target the velocity at which third phase guidance, such as heading alignment \cite{MSL_EDL2} or final position alignment \cite{GuangfeiDissertation}, begins. 

Discussion of parachute conditions that we aren't modeling in this phase. The parachute conditions could be included in the assessment conditions. Discussion of parachute-free possibilities? 

% The topic of this section is to discuss the actual entry guidance requirements and problem. Not necessarily to pose the robust OCP (?)

%The robust optimal guidance problem is to determine the reference control $u_{\mathrm{ref}}$ that minimizes the cost functional
%\begin{align}
%	J(x,u) = -\E{h} + w_h\sigma_h + w_s\sigma_s \label{eq:cost:mayer}
%\end{align}
%subject to the dynamics
%\begin{align}
%	\dot{\state}(t,\param) = \mathbf{f}(\state(t,\param), u(t), \param)
%\end{align}
%with the initial conditions $\state(0,\param) = \state_0(\param)$, the terminal constraint $v(t_f,\param)=v_f$, and the control constraint $u_{\min}\le u_{\mathrm{ref}}(t)\le u_{\max}\,\forall\,t\in[0,t_f]$. Let $\Omega$ be the space of stochastic parameters with probability density function $\mu(\param)$. The expectation and standard deviations are computed with respect to the uncertain parameters, i.e.,  $\E{F(\state,\param)} = \int_{\Omega}\,F(\state,\param)\mu(\param)\,\mathrm{d}\param$ and $\sigma^2_{F(\state,\param)} = \int_{\Omega}\,\left(F(\state,\param)-\E{F(\state,\param)}\right)^2\mu(\param)\,\mathrm{d}\param$.
%\section{}


%%% Local Variables: ***
%%% mode: latex ***
%%% TeX-master: "thesis.tex" ***
%%% End: ***
